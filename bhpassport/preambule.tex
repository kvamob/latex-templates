% Этот шаблон документа разработан в 2017 году
% Владимиром Коротковым (kvamob@mail.ru) 
% для подготовки и печати паспортов скважин 

\documentclass[a4paper,12pt]{article} % добавить leqno в [] для нумерации слева

%% Глобальные Параметры страницы
\usepackage[left=3cm,right=2cm,
top=1cm,bottom=2cm,bindingoffset=0cm]{geometry}

\usepackage{fp} 						% Вычисления с плавающей точкой
\usepackage{siunitx}					% При использовании пакета fp все числа должны иметь decimal delimiter точку
\sisetup{output-decimal-marker={,}}		% Числа выводятся с запятой в качестве разделителя разрядов: \num{3.2} выводит 3,2 

\usepackage{etoolbox} 					% логические операторы

%%% Сменим шрифт
%\renewcommand{\rmdefault}{ptm}
%\renewcommand{\rmdefault}{cmss}
%\renewcommand{\familydefault}{\sfdefault} 

%%% Работа с русским языком
\usepackage{cmap}					% поиск в PDF
\usepackage{mathtext} 				% русские буквы в формулах
\usepackage[T2A]{fontenc}			% кодировка
\usepackage[utf8]{inputenc}			% кодировка исходного текста
\usepackage[english,russian]{babel}	% локализация и переносы

%%% 
\usepackage{rotating}				% Поворот текста

\usepackage{fancyhdr}				% Работа с колонтитулами

%%% Дополнительная работа с математикой
\usepackage{amsmath,amsfonts,amssymb,amsthm,mathtools} % AMS
\usepackage{icomma} % "Умная" запятая: $0,2$ --- число, $0, 2$ --- перечисление

%%% Символ диаметра
\DeclareFontEncoding{LS1}{}{}
\DeclareFontSubstitution{LS1}{stix}{m}{n}
\DeclareRobustCommand{\diameter}{%
	\text{\usefont{LS1}{stixscr}{m}{n}\symbol{"60}}%
}

%% Номера формул
\mathtoolsset{showonlyrefs=true} % Показывать номера только у тех формул, на которые есть \eqref{} в  тексте.

%% Шрифты
%\usepackage{euscript}	 % Шрифт Евклид
%\usepackage{mathrsfs} % Красивый матшрифт

%% Свои команды
% \DeclareMathOperator{\sgn}{\mathop{sgn}}

%% Перенос знаков в формулах (по Львовскому)
% \newcommand*{\hm}[1]{#1\nobreak\discretionary{}
%	{\hbox{$\mathsurround=0pt #1$}}{}}

%%% Работа с картинками
\usepackage{graphicx}  % Для вставки рисунков
%\usepackage[export]{adjustbox}
\graphicspath{{images/}{images2/}}  % папки с картинками
\setlength\fboxsep{3pt} % Отступ рамки \fbox{} от рисунка
\setlength\fboxrule{0.2pt} % Толщина линий рамки \fbox{}
\usepackage{wrapfig} % Обтекание рисунков и таблиц текстом

%%% Работа с таблицами
\usepackage{array,tabularx,tabulary,booktabs} % Дополнительная работа с таблицами
\usepackage{longtable}  % Длинные таблицы
\usepackage{multirow} % Слияние строк в таблице

%%% Подписи к рисункам и таблицам в русской типографской традиции
\usepackage{caption} 
\DeclareCaptionFormat{GOSTtable}{#2#1\\#3}
\DeclareCaptionLabelSeparator{fill}{\hfill}
\DeclareCaptionLabelSeparator{dot}{. }
\DeclareCaptionLabelFormat{fullparents}{\bothIfFirst{#1}{~}#2}
\captionsetup[table]{
	format=GOSTtable,
	font={footnotesize},
	labelformat=fullparents,
	labelsep=fill,
	labelfont=rm,
	%	labelfont=it,
	textfont=bf,
	justification=centering,
	singlelinecheck=false
}
\captionsetup{font=small}
\captionsetup[figure]{
	labelsep=dot, 
	%	textfont=it
}
% А можно и так
%\captionsetup{labelsep=period}


%%% Модификация команд, задающих разделы
% Не подавлять отступы у первого абзаца

\makeatletter   % Команда \makeatletter делает символ @ буквой, команда \makeatother возвращает всё на свои места.
% Разрешим отступ у первого абзаца
\renewcommand\section{\@startsection {section}{1}{\parindent}%
	{3.5ex \@plus 1ex \@minus .2ex}{2.3ex \@plus.2ex}%
	{\normalfont\hyphenpenalty=10000\Large\bfseries}}

\renewcommand\subsection{\@startsection {subsection}{1}{\parindent}%
	{3.5ex \@plus 1ex \@minus .2ex}{2.3ex \@plus.2ex}%
	{\normalfont\hyphenpenalty=10000\large\bfseries}}
\makeatother

% После номеров разделов \section ставить точки
\usepackage{secdot}			
% И после \subsection тоже ставить точки
\sectiondot{subsection}		

\usepackage{setspace} % Интерлиньяж
%\onehalfspacing % Интерлиньяж 1.5
%\doublespacing % Интерлиньяж 2
%\singlespacing % Интерлиньяж 1

\usepackage{lastpage} % Узнать, сколько всего страниц в документе.

\usepackage{soulutf8} % Модификаторы начертания

%\usepackage{hyperref}
%\usepackage[usenames,dvipsnames,svgnames,table,rgb]{xcolor}
%\hypersetup{				% Гиперссылки
%	unicode=true,           % русские буквы в раздела PDF
%	pdftitle={Заголовок},   % Заголовок
%	pdfauthor={Автор},      % Автор
%	pdfsubject={Тема},      % Тема
%	pdfcreator={Создатель}, % Создатель
%	pdfproducer={Производитель}, % Производитель
%	pdfkeywords={keyword1} {key2} {key3}, % Ключевые слова
%	colorlinks=true,       	% false: ссылки в рамках; true: цветные ссылки
%	linkcolor=red,          % внутренние ссылки
%	citecolor=green,        % на библиографию
%	filecolor=magenta,      % на файлы
%	urlcolor=cyan           % на URL
%}

\usepackage{multicol} % Несколько колонок

\usepackage{fancyhdr} % Колонтитулы
%\pagestyle{fancy}
%\renewcommand{\headrulewidth}{0mm}  % Толщина линейки, отчеркивающей верхний колонтитул
%\lfoot{Нижний левый}
%\rfoot{Нижний правый}
%\rhead{Верхний правый}
%\chead{Верхний в центре}
%\lhead{Верхний левый}
% \cfoot{Нижний в центре} % По умолчанию здесь номер страницы
