% Этот шаблон документа разработан в 2017 году
% Владимиром Коротковым (kvamob@mail.ru) 
% для подготовки и печати паспортов скважин 

\documentclass[a4paper,12pt]{article} % добавить leqno в [] для нумерации слева

%%% Работа с русским языком
\usepackage{cmap}					% поиск в PDF
% \usepackage{mathtext} 				% русские буквы в формулах
\usepackage[T2A]{fontenc}			% кодировка
\usepackage[utf8]{inputenc}			% кодировка исходного текста
\usepackage[english,russian]{babel}	% локализация и переносы

%%% Дополнительная работа с математикой
% \usepackage{amsmath,amsfonts,amssymb,amsthm,mathtools} % AMS
% \usepackage{icomma} % "Умная" запятая: $0,2$ --- число, $0, 2$ --- перечисление

%% Номера формул
% \mathtoolsset{showonlyrefs=true} % Показывать номера только у тех формул, на которые есть \eqref{} в  тексте.

%% Шрифты
% \usepackage{euscript}	 % Шрифт Евклид
% \usepackage{mathrsfs} % Красивый матшрифт

%% Свои команды
% \DeclareMathOperator{\sgn}{\mathop{sgn}}

%% Перенос знаков в формулах (по Львовскому)
% \newcommand*{\hm}[1]{#1\nobreak\discretionary{}
%	{\hbox{$\mathsurround=0pt #1$}}{}}

%%% Заголовок
\author{ООО «Гидросфера»}\label{company}
\title{ПАСПОРТ РАЗВЕДОЧНО-ЭКСПЛУАТАЦИОННОЙ СКВАЖИНЫ}
\date{\today}

\newcommand{\txtExecutor}{ООО «Гидросфера»}
\newcommand{\txtAddress}{Свердловская обл, г. Екатеринбург, ул. Кутузова, дом 47}
\newcommand{\txtCadaster}{66:41:0508041:8}
%% ../ приуроченных к ...
\newcommand{\txtGeology}{трещиноватым разностям палеозоя}


\begin{document} % конец преамбулы, начало документа

%% Титульная страница

\begin{titlepage}
	\begin{center}
		\textbf{\txtExecutor}
		\vspace{5.5cm}
		
		{\LARGE ПАСПОРТ РАЗВЕДОЧНО-ЭКСПЛУАТАЦИОННОЙ СКВАЖИНЫ}
		\vspace{0.25cm}
		
		\underline{для хозяйственно-бытового водоснабжения}
		
		\bigskip
		
		на участке по адресу:
				
		\underline{\txtAddress}
		
		Кадастровый номер \txtCadaster
		
		\vfill
	
		\bigskip
		
	\end{center}

	\vfill
	
	\newlength{\ML}
	\settowidth{\ML}{«\underline{\hspace{0.7cm}}» \underline{\hspace{2cm}}}
	\hfill\begin{minipage}{1.0\textwidth}
		Директор ООО «Гидросфера» к.г.м.н.
		\underline{\hspace{\ML}} А.\,А.~Кашкаров\\
	\end{minipage}%
	\bigskip
	
%		Ответственный исполнитель, к.г.м.н.
%		\underline{\hspace{\ML}} А.\,А.~Кашкаров\\
%		«\underline{\hspace{0.7cm}}» \underline{\hspace{2cm}} 2014 г.
%	\end{minipage}%

	\vfill
	
	\begin{center}
		Екатеринбург, 2017 г.
	\end{center}
\end{titlepage}


	\maketitle
	
	Адрес участка 

	\begin{enumerate}
		\item Формулировка проблемы
		\item Определение предмета исследования
		\item Определение цели исследования
		\item Постановка задач исследования
		\item Установка ограничений
		\item Определение необходимой информации
		\item Выявление объектов исследования
	\end{enumerate}
		
	\section{Расположение скважины}
	
	\subsection{Дроби}
	
	\subsection{Скобки}
	
	\subsection{Стандартные функции}
	
	\subsection{Символы}
	
	\subsection{Диакритические знаки}
	
	\subsection{Буквы других алфавитов}
	
\end{document} % конец документа

