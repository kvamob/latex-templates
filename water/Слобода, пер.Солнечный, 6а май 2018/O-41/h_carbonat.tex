\textbf{Трещинные воды}. Уральская сложная гидрогеологическая складчатая область располагается в~пределах орографически выраженного одноименного горно - складчатого сооружения. Основными коллекторами подземных вод являются трещиноватые породы коренного субстрата. Мощность зоны региональной трещиноватости составляет в~среднем 30  --  100~м. Минимальные значения (20  --  30~м) присущи корам выветривания интрузивных пород; максимальные (60  --  100~м)  --  карбонатным породам; средние (40  --  60~м)  ---  эффузивно-осадочным и~метаморфическим комплексам. Помимо трещин выветривания широко развиты локальные линейные трещинные зоны высокой проницаемости и~водоотдачи, связанные с проявлениями дизъюнктивной тектоники и~контактами разнородных пород. Подземные воды региональной трещиноватости обычно гидравлически взаимосвязаны, имеют безнапорный характер и~образуют небольшие бассейны с интенсивным
водообменном. В вертикальном разрезе фильтрационные свойства пород зоны выветривания неоднородны. По характеру их изменения зона разделяется на~три части. В верхней (10  --  20~м), где широко представлены глины или суглинки элювиальной коры выветривания, водопроницаемые свойства очень низки; особенно широко коры распространены на~площади Зауральского пенеплена. Средняя часть эрозионной зоны отличается наибольшей активностью, сильной степенью трещиноватости и~высокой пористостью (от 1 до~7~\%). В нижней части размеры трещин весьма незначительны, и~водоотдача пород практически отсутствует.

\textit{Водоносная зона трещиноватости средне-верхнепалеозойских карбонатных образований} ($сPZ_{2–3}$) является одной из~наиболее водообильных. Она развита преимущественно в~пределах синклинальных, реже антиклинальных структурных форм в~виде отдельных разобщенных либо взаимосвязанных между собой водоносных горизонтов меридионального и~субмеридионального простирания. Наиболее изученные Невьянская, Алапаевская,
Каменская зоны имеют площади до~200 км\textsuperscript{2}. Водовмещающими породами являются известняки с пачками и~прослоями глинистых сланцев, аргиллитов, песчаников и~туффитов. Трещиновато-карстовая зона имеет мощность 50  --  80~м, достигая в~зонах тектонических нарушений 200  --  250~м. С поверхности карбонатные породы осложнены проявлениями карста (в виде воронок); в~пределах мезозойских депрессий карст перекрыт кайнозойскимиотложениями, мощность до~20  --  30~м.

Карстовым процессам подвержены все карбонатные <<массивы>>, но степень их проявления неравномерна. На Алапаевском «массиве» скважинами выявлены погребенные щелевидные депрессии длиной до~900  --  2000~м
при ширине 400  --  500~м и~глубиной от 70 до~140~м в~осевой части.

Характерной особенностью древних карстовых депрессий является высокая трещиноватость бортовых частей и~слабая водонасыщенность днищ. Погребенные карстово - трещинные воды в~депрессиях обладают напором, соответствующим мощности экранирующего покрова. Питание подземных вод осуществляется за счет инфильтрации атмосферных осадков и~разгрузки сопряженных вод из~других горизонтов. Циркуляция происходит по~сложному лабиринту карстовых пустот и~трещин, коэффициент фильтрации в~которых варьирует от 2  --  5 до~30~м/сут. Максимальная водообильность приурочена к~придолинным участкам пересечения с линейными водоносными зонами. Дебиты родников в~них достигают 10  --  25~л/с; в~стороне от речных долин вне водоносных зон дебиты не превышают 3  --  5~л/с. Химический состав и~минерализация трещинно-карстовых вод изменяются в~меридиональном направлении: до~широты г. Алапаевск преобладают гидрокарбонатные кальциевые, реже кальциево-магниевые воды с минерализацией от 0,1 до~0,4 г/дм\textsuperscript{3}; южнее распространены гидрокарбонатные кальциево-магниевые, реже кальциевые воды с минерализацией от 0,2 до~0,6 -- 0,8 г/дм\textsuperscript{3}.

В целом питание подземных вод Большеуральского бассейна сезонное, за счет инфильтрации атмосферных осадков в~теплый период года. Зеркало его вод в~сглаженной форме повторяет основные элементы рельефа. На склонах и~уплощенных водоразделах уровни воды залегают на~глубинах 5  --  20~м, на~хребтах и~локальных возвышенностях  ---  до~30  --  50~м. Сравнительно глубокая расчлененность дневной поверхности, особенно в~районах приподнятых горных массивов, обеспечивает хорошие условия дренирования водоносных зон речной сетью. Разгрузка вод идет преимущественно вдоль долин рек, а также может быть приурочена к~локальным трещинным зонам. Дебиты родников, в~зависимости от величины водосборной площади, варьируют от долей до~многих десятков литров в~секунду.

Эксплуатационные ресурсы Большеуральского бассейна связываются преимущественно с крупными карбонатными массивами среднего и~верхнего палеозоя и~тектонически активными зонами разломов, на~которых возможна организация водозаборов с дебитом 100  --  1000~л/с. На остальных водоносных зонах трещиноватости возможен каптаж подземных вод по~отдельным кустам скважин с дебитами от 10 до~30~л/с. Ресурсы порово-пластовых вод Иртыш-Обского бассейна связаны преимущественно с опоковым горизонтом палеоцена, мощность обводненной толщи которого 3  --  4~м на~западе и~до~50~м на~востоке. На придолинных участках некоторые водозаборы из~него имеют производительность до~50 -- 100~л/с.

Несмотря на~значительное количество водоносных горизонтов на~изученной площади, наиболее промышленные и~обжитые районы Урала испытывают недостаток в~водообеспеченности из-за неравномерного распределения ресурсов, а также в~связи с невозможностью создания крупных водозаборов с высокой производительностью на~локальных участках водоносных зон.
