% Этот шаблон документа разработан в 2017 году
% Владимиром Коротковым (kvamob@mail.ru) 
%  

\documentclass[a4paper,12pt]{article} % добавить leqno в [] для нумерации слева

%% Глобальные Параметры страницы
\usepackage[left=3cm,right=2cm,top=1cm,bottom=2cm,bindingoffset=0cm]{geometry}

%\usepackage{fp} 						% Вычисления с плавающей точкой
%\usepackage{siunitx}					% При использовании пакета fp все числа должны иметь decimal delimiter точку
%\sisetup{output-decimal-marker={,}}	% Числа выводятся с запятой в качестве разделителя разрядов: \num{3.2} выводит 3,2 

%%% Работа с русским языком
\usepackage{cmap}					% поиск в PDF
\usepackage{mathtext} 				% русские буквы в формулах
\usepackage[T2A]{fontenc}			% кодировка
\usepackage[utf8]{inputenc}			% кодировка исходного текста
\usepackage[english,russian]{babel}	% локализация и переносы

%%% 
%\usepackage{rotating}				% Поворот текста

%%% Дополнительная работа с математикой
\usepackage{amsmath,amsfonts,amssymb,amsthm,mathtools} % AMS
\usepackage{icomma} % "Умная" запятая: $0,2$ --- число, $0, 2$ --- перечисление
\usepackage{gensymb}	% Градусы
%% Номера формул
%\mathtoolsset{showonlyrefs=true} % Показывать номера только у тех формул, на которые есть \eqref{} в  тексте.

%% Шрифты
\usepackage{euscript}	 % Шрифт Евклид
\usepackage{mathrsfs} % Красивый матшрифт


%% Перенос знаков в формулах (по Львовскому)
% \newcommand*{\hm}[1]{#1\nobreak\discretionary{}
%	{\hbox{$\mathsurround=0pt #1$}}{}}

%%% Работа с картинками
\usepackage{graphicx}  % Для вставки рисунков
%\usepackage[export]{adjustbox}
\graphicspath{{images/}}  % папки с картинками
\setlength\fboxsep{3pt} % Отступ рамки \fbox{} от рисунка
\setlength\fboxrule{0.2pt} % Толщина линий рамки \fbox{}
\usepackage{wrapfig} % Обтекание рисунков и таблиц текстом


%%% Работа с таблицами
\usepackage{array,tabularx,tabulary,booktabs} % Дополнительная работа с таблицами
\usepackage{longtable}  % Длинные таблицы
\usepackage{multirow} % Слияние строк в таблице

%%% Подписи к рисункам и таблицам в русской типографской традиции
\usepackage{caption} 
\DeclareCaptionFormat{GOSTtable}{#2#1\\#3}
\DeclareCaptionLabelSeparator{fill}{\hfill}
\DeclareCaptionLabelSeparator{dot}{. }
\DeclareCaptionLabelFormat{fullparents}{\bothIfFirst{#1}{~}#2}
\captionsetup[table]{
	format=GOSTtable,
	font={footnotesize},
	labelformat=fullparents,
	labelsep=fill,
	labelfont=rm,
%	labelfont=it,
	textfont=bf,
	justification=centering,
	singlelinecheck=false
}
\captionsetup{font=small}
\captionsetup[figure]{
	labelsep=dot, 
%	textfont=it
}
% А можно и так
%\captionsetup{labelsep=period}


%%% Модификация команд, задающих разделы
% Не подавлять отступы у первого абзаца

\makeatletter   % Команда \makeatletter делает символ @ буквой, команда \makeatother возвращает всё на свои места.
% Разрешим отступ у первого абзаца
\renewcommand\section{\@startsection {section}{1}{\parindent}%
	{3.5ex \@plus 1ex \@minus .2ex}{2.3ex \@plus.2ex}%
	{\normalfont\hyphenpenalty=10000\Large\bfseries}}

\renewcommand\subsection{\@startsection {subsection}{1}{\parindent}%
	{3.5ex \@plus 1ex \@minus .2ex}{2.3ex \@plus.2ex}%
	{\normalfont\hyphenpenalty=10000\large\bfseries}}
\makeatother

% После номеров разделов \section ставить точки
\usepackage{secdot}			
% И после \subsection тоже ставить точки
\sectiondot{subsection}		


%%%%%%%%%%%%%%%%%%%%%%%%%%%%%%%%%%%%%%%%%%%%%%%%%%%%%%%%%%%%%%%%%%%%%%%%%%%%%%%%%%%%%%%%%%%%%%%%%%%%%%%%%
%%% PAYLOAD
%%%%%%%%%%%%%%%%%%%%%%%%%%%%%%%%%%%%%%%%%%%%%%%%%%%%%%%%%%%%%%%%%%%%%%%%%%%%%%%%%%%%%%%%%%%%%%%%%%%%%%%%%

%%% Заголовок
\author{ООО <<Гидросфера>>}\label{company}
\title{ОТЧЕТ ПО РЕЗУЛЬТАТАМ ПОИСКОВ ИСТОЧНИКОВ ПОДЗЕМНЫХ ВОД}
\date{\today}
%%%======================================================================================================
\newcommand{\txtExecutor}{ООО <<Гидросфера>>}	% Исполнитель
\newcommand{\txtYear}{2017}						% Год
\newcommand{\txtAddress}{--Address--}			% Адрес
\newcommand{\txtCadaster}{--Cadaster--} 		% Кадастровый номер


%%%%%%%%%%%%%%%%%%%%%%%%%%%%%%%%%%%%%%%%%%%%%%%%%%%%%%%%%%%%%%%%%%%%%%%%%%%%%%%%%%%%%%%%%%%%%%%%%%%%%%%%%

\begin{document} % конец преамбулы, начало документа

\section*{ЛИСТ O-41}
\section*{ГИДРОГЕОЛОГИЯ}
стр.371 в пояснительной записке к карте 0-41

В соответствии с гидрогеологическим районированием Урала и перечнем бассейнов подземных вод (ВСЕГИНГЕО, 2004) на площади листа О-41 выделяются два надпорядковых геологических таксона: Уральская сложная гидрогеологически складчатая область корово-блоковых вод и Западно-Сибирский сложный артезианский бассейн. Первая приурочена к открытым выходам на дневную поверхность комплекса пород Уральской складчатой системы в геоморфологических границах Уральского горного сооружения. Второй представлен западной частью Иртыш-Обского артезианского бассейна пластовых вод второго порядка и связан с морскими мезозойско-кайнозойскими отложениями Западно-Сибирской равнины. 
Граница между ними проходит по контакту палеозойских и мезокайнозойских морских толщ.

\section*{ГИДРОГЕОЛОГИЧЕСКИЕ ПОДРАЗДЕЛЕНИЯ}

\section*{Западно-Сибирский артезианский бассейн}

Порово-пластовые и трещинно-пластовые воды. Подземные воды, связанные с неоген-четвертичными образованиями, распространены на площади всего листа и связаны с субаквальными генетическими типами.

\textit{Голоценовый озерно-палюстринный водоносный горизонт} ($lplQ_н$) распространен на северо-востоке площади. Он представлен иловатыми алевритами, перекрытыми торфом; мощность до 10 м. Осадки насыщены водой на всю мощность. Коэффициент фильтрации торфов 0,17 – 2,5, илов и иловатых
алевритов – 0,15 – 0,5 м/сут. Воды пресные, обогащенные гумидными кислотами.

\textit{ Верхненеоплейстоценово - голоценовый аллювиальный водоносный горизонт } \\($aQ_{III–Н}$) представлен русловыми фациями аккумулятивных камышловской, режевской надпойменных террас, а также поймы. Фации сложены гравийными песками с галькой в верховьях рек и разнозернистыми песками – в среднем и нижнем течении. Они формируют нижние слои террасовых разрезов, залегая на близких гипсометрических уровнях; их водоносные слои сопряжены между собой. Мощности русловых фаций от 3 – 5 до 10 м. Коэффициенты фильтрации гравийных песков – 7 – 25, пылеватых песков – 5 – 13, супесей – 1,5 – 3,2 м/сут. Дебиты скважин варьируют от 0,2 до 2 – 6 л/с. Питание осуществляется за счет атмосферных осадков, поверхностных паводковых вод и разгрузки в долины рек подземных вод из пород,слагающих борта долин. По химическому составу воды гидрокарбонатные кальциевые с минерализацией до 0,5 г/дм\textsuperscript{3}.

\textit{Средне-верхненеоплейстоценовый озерно - аллювиальный водоносный горизонт} ($laQ_{II–III}$) приурочен к озерно-аллювиальной четвертой надпойменной террасе рек Лозьва, Пелым, Конда и представлен переслаиванием песков разнозернистых, алевритов и синевато-серых глин, залегающих в нижней части разреза. Мощность горизонта до 20 м. Коэффициент фильтрации водовмещающих отложений варьирует от 0,5 до 3,0 м/сут. Глубина залегания  грунтовых вод до 11 м. Воды безнапорные, водообильность незначительная. 
По химическому составу воды гидрокарбонатные, по степени минерализации (до 0,5 г/дм\textsuperscript{3}) – ультрапресные и пресные. Питание за счет инфильтрации атмосферных осадков и подтока вод из вышезалегающих горизонтов. Разгрузка происходит в долины рек.

\textit{Средненеоплейстоценовый озерный и озерно-аллювиальный водоносный горизонт} ($l, laQ_{II}$) приурочен к озерно-аллювиальным отложениями высоких надпойменных террас рек Тура, Пышма, Исеть, а также к озерным, 
озерно-аллювиальным осадкам ледниково-подпрудного бассейна на северо-востоке площади в долинах рек Конда и Тавда. Водовмещающими породами являются пылеватые пески и алевриты (с прослоями глин) и разнозернистые пески в основании. Мощность 20 – 30 м. Водоупором служат глины тавдинской свиты на западе и туртасской свиты на северо-востоке. Питание водоносного горизонта происходит путем инфильтрации атмосферных осадков, на западе – частично за счет подтока вод из вышезалегающих горизонтов. Воды безнапорные, водообильность незначительная. Дебиты родников в бортах долин не превышают 0,5 л/с. Воды пресные, с минерализацией до 0,9 г/дм\textsuperscript{3}, по составу гидрокарбонатные. В долине Туры на участках разгрузки более древних водоносных горизонтов воды приобретают хлоридный или хлоридно-гидрокарбонатный натриевый состав.

\textit{Неоген-четвертичный аллювиальный и озерный водоносный горизонт} ($a, lN–Q$) имеет широкое распространение на междуречьях речных долин.
Он связан с отложениями наурзумской, пелымской, светлинской, кустанайской, батуринской свит и падунского озерного комплекса, выполняющими палеоложбины и древние озерные ванны. Осадки представлены песками, алевролитами и глинами. Водоносными являются слои песков мощностью до 1,2 м, выстилающих ложе палеопонижений и залегающих на водоупорных породах. Питание горизонта происходит преимущественно за счет атмосферных осадков. Дебиты отдельных скважин и колодцев довольно низкие (до 0,5 л/с). Качество вод весьма «пестрое» – от пресных до соленых, с минерализацией от 0,5 до 9,5 г/дм\textsuperscript{3}.

\textit{Верхнеолигоценовый (туртасский) относительно водоупорный горизонт} ($P_3tr$) распространен в северо-восточной части листа и связан с осадками туртасской свиты. Они сложены глинами с прослоями алевритов; мощность их 5 – 10 м. В основании иногда отмечаются алевритистые водонасыщенные пески (0,2 – 0,5 м). Дебиты отдельных скважин варьируют от 0,1 до 0,5 л/с. Воды пресные, гидрокарбонатные кальциево-магниевые, с минерализацией до 0,5 г/дм\textsuperscript{3}.

\textit{Нижнеолигоценовый (куртамышский) водоносный горизонт} ($P_3kr$) развит преимущественно на водоразделах. 
Водовмещающими являются пески континентальных и прибрежно-морских отложений (куртамышская, атлымская, новомихайловская свиты), перекрытые маломощной толщей субаэральных образований. 
Наибольшее значение среди них имеют куртамышские пески мощностью до 20 – 30 м. 
Эрозионная расчлененность горизонта обусловила разделение его на многочисленные относительно разобщенные
бассейны подземных вод грунтового характера с присущими им областями питания, стока и гидрохимического режима. Водообильность песчаных отложений небольшая и весьма неравномерная. Дебиты скважин изменяются в пределах 0,5 – 1,5 л/с. Средняя величина коэффициента фильтрации песков 2 – 5 м/сут. Подземные воды пресные, преимущественно гидрокарбонатного кальциевого состава.

\textit{Эоценовый (ирбитско-тавдинский) водоупорный региональный горизонт} ($P_2ir–tv$) 
имеет широкое распространение в Зауралье. Его формируют ирбитская свита, сложенная диатомитами и диатомитовыми глинами, и тавдинская свита, состоящая из листоватых монтмориллонитовых глин. Общая их мощность – более 200 м. 
Нижняя (ирбитская) свита служит экраном для напорного серовского горизонта; верхняя (тавдинская) –
основанием куртамышского.

\textit{Палеоценовый опоковый (серовский) водоносный горизонт} ($P_1sr$) приурочен 
к морским отложениям серовской свиты, которая сложена преимущественно кремнистыми и 
глинистыми опоками. В западной части артезианского бассейна кремнистые породы серовской 
свиты иногда замещаются кварц-глауконитовыми песчаниками с опоковым цементом, на которых
залегают кремнистые серые опоки с включениями песчаного материала.
По мере погружения кровли серовской свиты на восток в составе опок широкое распространение получают глинистые разности при подчиненном значении кремнистых, которые еще далее на восток замещаются алевролитами и аргиллитами с редкими прослоями опок. Мощность изменяется от 20 до 80 м при средних значениях 40 – 60 м. Переход серовской свиты к вышележащим отложениям постепенный, в виде переслаивания опок с диатомитами и диатомитовыми глинами нижней части разреза эоценовой
ирбитской свиты. Нижняя граница водоносных серовских отложений проводится по кровле глин талицкой свиты нижнего палеоцена. «Опоковый» горизонт представляет большой интерес как источник хозяйственно-питьевого водоснабжения.

В восточном направлении, с появлением в гидрогеологическом разрезе слабопроницаемых (коэффициенты фильтрации 0,01 – 0,09 м/сут) диатомитов и диатомитовых глин ирбитской водоупорной свиты (см. выше), 
горизонт приобретает напорный режим. Величина напора в долинах рек колеблется от первых метров до 10–20 м, на водоразделах достигает 40 – 70 м. 
Водообильность опокового горизонта крайне неравномерна. Удельные дебиты изменяются от тысячных долей до 26,0 л/с, при преобладающих 1,0 – 4,0 л/с. Повышенная водоносность опок наблюдается над зонами региональных глубинных разломов. Примером служит скважина в 13 км к юго-востоку от ж./д. ст. Сосьва с дебитом при самоизливе из опок 43 л/с.
Химический состав подземных вод также претерпевает изменения в восточном направлении; это связанно с ростом мощности перекрывающего его водоупора и сменой пресных гидрокарбонатных вод на солоноватые и соленые сульфатного и хлоридного составов. Минерализация подземных вод не превышает 1 г/дм\textsuperscript{3}; при мощности слабопроницаемой толщи от 40 до 80 м она возрастает до 3 г/дм\textsuperscript{3}; при мощности более 80 м – превышает 3 г/дм\textsuperscript{3}. По мере роста минерализации появляются микрокомпоненты, в первую очередь бор и бром, концентрация которых (при минерализации более 1 г/дм\textsuperscript{3}) значительно превышает ПДК, установленные для питьевых вод.

\textit{Верхнемеловой–эоценовый водоносный горизонт} ($К_2–P_2$) приурочен к крайней западной 
части Иртыш-Обского бассейна, где фрагментарно развиты меловые и палеогеновые отложения. 
В депрессионных зонах на меловых песках залегают трещиноватые опоки и песчаники на опоковом цементе 
(серовская свита), перекрытые ирбитскими диатомитами либо более поздними образованиями. 
Мощность водонасыщенных песков и опок 10 – 20 м.
Водообильность возрастает вдоль тектонических нарушений в связи с повышением трещиноватости опокового слоя. Подземные воды имеют свободный уровень. Питание их осуществляется на склонах речных долин и водораздельных пространствах путем инфильтрации атмосферных осадков.
К этой же зоне относится Каменск-Уральская группа месторождений под-земных вод, 
а также водозаборы близ г. Алапаевск. В глубоких мезозойских эрозионно-тектонических депрессиях, таких как Черноскутовская, Ялунинская, водозаборы имеют производительность до 72 л/с. Дебиты скважин весьма изменчивы, преобладают 1 – 3 л/с. Разгрузка происходит в виде родникового стока в долинах крупных рек. Доминируют воды с минерализацией до 0,5 г/дм\textsuperscript{3}, гидрокарбонатного или сульфатно-гидрокарбонатного магниево-кальциевого, реже смешанного катионного состава.

\textit{Верхнемеловой–палеоценовый (ганькинско-талицкий) региональный водоупорный горизонт} ($К_2gn–P_1tl$) образуют ганькинская (глины алевритистые) и талицкая (глины монтмориллонитовые с прослоями алевролитов) свиты. Общая мощность свыше 200 м. Ганькинская свита является экраном верхнемелового, талицкая – водоупором вышеописанного опокового горизонтов.

\textit{Верхнемеловой водоносный горизонт} ($К_2$) имеет повсеместное развитие. Он приурочен к отложениям камышловской и зайковской свит. Гидрогеологические условия водоносного горизонта связаны с литолого-фациальными особенностями строения водовмещающих пород. В западной части
бассейна они отвечают тонкозернистым кварц-глауконитовым пескам с подчиненными прослоями глин. Мощность водовмещающих пород варьирует от первых метров до 60 м. По химическому составу подземные воды
на западе бассейна пресные гидрокарбонатные с переменным, преимущественно натриевым катионным составом. Минерализация воды до 1,0 г/дм\textsuperscript{3} распространена на площади примерно ограниченной линией, соединяющей на водоразделах западные фланги развития диатомитов ирбитской свиты. Далее на восток, по мере увеличения мощности покровных отложений, она постепенно увеличивается с 1,5–3,0 г/дм\textsuperscript{3} (г. Камышлов) до
5–10 г/дм\textsuperscript{3} и более (г. Талица). В восточной части бассейна водовмещающие породы представлены глауконит-кварцевыми песчаниками с глинистым цементом, на крайнем востоке – алевролитами. Граница между преобладающими фациями в разрезе водовмещающих пород ориентирована субмеридионально и проходит восточнее г. Талица. Нижним водоупором горизонта являются глины кузнецовской свиты. Водообильность горизонта
весьма различна. Минерализация подземных вод повсеместно превышает 3 г/дм\textsuperscript{3}. В Тавдинском районе водоносный горизонт связан с кварцевыми песчаниками и алевропесчаниками уватской свиты. Дебиты по редким скважинам изменяются от 0,3 до 2,47 л/с. Воды соленые, с минерализацией 12–13 г/дм\textsuperscript{3}, хлоридные натриевые, с микрокомпонентами брома – 27 мг/дм\textsuperscript{3} и йода – 3 мг/дм\textsuperscript{3}. Газовый состав метановый, воды термальные (33–40 \degreeС).

\textit{Нижнемеловой водоносный комплекс} ($К_1$) связан с морскими ахской, леушинской, кошайской, викуловской и ханты-мансийской свитами Восточного Зауралья, погребенными под осадками верхнего мела и палеогена.
Свиты представлены неравномерным переслаиванием алевролитов, аргиллитов, песчаников, песков и глин с подчиненным развитием глинистых известняков и сидерититов. В этой мощной толще (до 700 м) выделяются
2–3 водонасыщенных горизонта, разделенных относительными водоупорами из глин, аргиллитов и алевролитов. Они приурочены к верхней части разрезов ханты-мансийской, викуловской и леушинской свит. Водонасыщенными (в разной степени) являются прослои песков, песчаников и глинистых известняков. В условиях глубокого залегания подземные воды насыщены газом, преимущественно метаном, имеют повышенную температуру (до 57 \degreeС); почти все скважины фонтанируют. Дебит скважин при самоизливе изменяется от незначительного до 80 л/с (Туринская опорная скважина), а удельный дебит варьирует от 0,01 до 0,40 л/с. Минерализация
вод довольно высокая – 4,3 – 17,9 г/дм\textsuperscript{3}; ее возрастание происходит с запада на восток и северо-восток, а также от верхних горизонтов к нижним.В нижних горизонтах воды преимущественно хлоридные натриевые, типичные для зон замедленной циркуляции. В отдельных пластах отмечается повышенное содержание кальция, и воды преобретают хлоридный кальциево-натриевый состав. Из микрокомпонентов в составе подземных вод присутствуют йод (4,27 – 17,0 мг/дм\textsuperscript{3}) и бром (17,0 – 54,0 мг/дм\textsuperscript{3}).

\textit{Верхнеюрский (даниловский) водуопорный горизонт} ($J_3dn$) имеет локальное
развитие в восточной части площади. Представлен даниловской свитой аргиллитоподобных глин и битуминозных аргиллитов. Мощность 56 – 92 м.

\textit{Среднеюрский (тюменский) водоносный горизонт} ($J_2tm$) залегает на доюрском фундаменте, выполняя понижения его палеорельефа. Он представлен песчаниками, гравелитами и конгломератами (с глинистым или известковистым цементом) в переслаивании глинистыми алевролитами и глинами. Мощность горизонта до 200 м. Подземные воды приурочены к конгломератам, гравелитам и песчаникам нижней части разреза и сопряжены с трещинными водами палеозойского основания. Водообильность горизонта незначительна. Дебиты отдельных скважин, по данным Ю. П. Черепанова, достигают 8 – 10 л/с. Воды напорные, термальные (температура до 80 \degreeС). Минерализация высокая (17–26 г/дм\textsuperscript{3}), по химическому составу воды хлоридные натриевые, с содержанием брома до 67 мг/дм\textsuperscript{3}, йода до 26 мг/дм\textsuperscript{3}. Газовый состав вод – метановый (до 94 \%), газонасыщенность – до 500 см\textsuperscript{3}/л. Близкие характеристики имеют трещинные воды подстилающего палеозойского фундамента.

\textit{Верхнетриасово-нижнеюрский водоносный горизонт} ($T_3–J_1$) приурочен к западной и центральной части территории (Волчанская и Буланаш-Елкинская грабен-депрессии) Подземные воды связаны с континентальной угленосной челябинской серией, сложенной переслаивающимися конгломератами, песчаниками, алевролитами, аргиллитами с прослоями бурых углей; породы смяты в пологие брахискладки. Водоносными являются пласты конгломератов, песчаников и бурых углей. В крыльях брахисинклиналей подземные воды имеют свободный уровень; в осевой их части они приобретают напор, величина которого соответствует глубине залегания. Водоносность горизонта весьма неравномерна. Удельные дебиты скважин изменяются от 0,05 до 3,5 л/с; коэффициенты фильтрации от 0,02 до 7,9 м/сут (минимальные значения у аргиллитов и алевролитов). Питание горизонта происходит за счет трещинно-блоковых вод из подстилающих палеозойских пород, слагающих борта эрозионно-тектонических депрессий, а также путем разгрузки подземных вод из перекрывающих отложений (опоковый горизонт в Буланашскo-Елкинской депрессии). Наиболее обводненной является зона регионального выветривания угленосных отложений мощностью до 50–100 м. Подземные воды имеют небольшую минерализацию (0,25–0,6 г/дм\textsuperscript{3}) и гидрокарбонатный кальциево-магниевый состав.


\section*{Уральская сложная гидрогеологическая складчатая область}

\textbf{Трещинные воды}. Уральская сложная гидрогеологическая складчатая область распологается в пределах орографически выраженного одноименного горно - складчатого сооружения. Основными коллекторами подземных вод являются трещиноватые породы коренного субстрата. Мощность зоны региональной трещиноватости составляет в среднем 30 – 100 м. Минимальные значения (20 – 30 м) присущи корам выветривания интрузивных пород; максимальные (60 – 100 м) – карбонатным породам; средние (40 – 60 м) – эффузивно-осадочным и метаморфическим комплексам. Помимо трещин выветривания широко развиты локальные линейные трещинные зоны высокой проницаемости и водоотдачи, связанные с проявлениями дизъюнктивной тектоники и контактами разнородных пород. Подземные воды региональной трещиноватости обычно гидравлически взаимосвязаны, имеют безнапорный характер и образуют небольшие бассейны с интенсивным
водообменном. В вертикальном разрезе фильтрационные свойства пород зоны выветривания неоднородны. По характеру их изменения зона разделяется на три части. В верхней (10 – 20 м), где широко представлены глины или суглинки элювиальной коры выветривания, водопроницаемые свойства очень низки; особенно широко коры распространены на площади Зауральского пенеплена. Средняя часть эрозионной зоны отличается наибольшей активностью, сильной степенью трещиноватости и высокой пористостью (от 1 до 7 \%). В нижней части размеры трещин весьма незначительны, и водоотдача пород практически отсутствует.

\textit{Водоносная зона трещиноватости средне-верхнепалеозойских карбонатных образований} ($сPZ_{2–3}$) является одной из наиболее водообильных. Она развита преимущественно в пределах синклинальных, реже антиклинальных структурных форм в виде отдельных разобщенных либо взаимосвязанных между собой водоносных горизонтов меридионального и субмеридионального простирания. Наиболее изученные Невьянская, Алапаевская,
Каменская зоны имеют площади до 200 км\textsuperscript{2}. Водовмещающими породами являются известняки с пачками и прослоями глинистых сланцев, аргиллитов, песчаников и туффитов. Трещиновато-карстовая зона имеет мощность 50 – 80 м, достигая в зонах тектонических нарушений 200 – 250 м. С поверхности карбонатные породы осложнены проявлениями карста (в виде воронок); в пределах мезозойских депрессий карст перекрыт кайнозойскимиотложениями, мощность до 20 – 30 м.

Карстовым процессам подвержены все карбонатные «массивы», но степень их проявления неравномерна. На Алапаевском «массиве» скважинами выявлены погребенные щелевидные депрессии длиной до 900 – 2000 м
при ширине 400 – 500 м и глубиной от 70 до 140 м в осевой части.

Характерной особенностью древних карстовых депрессий является высокая трещиноватость бортовых частей и слабая водонасыщенность днищ. Погребенные карстово - трещинные воды в депрессиях обладают напором, соответствующим мощности экранирующего покрова. Питание подземных вод осуществляется за счет инфильтрации атмосферных осадков и разгрузки сопряженных вод из других горизонтов. Циркуляция происходит по сложному лабиринту карстовых пустот и трещин, коэффициент фильтрации в которых варьирует от 2 – 5 до 30 м/сут. Максимальная водообильность приурочена к придолинным участкам пересечения с линейными водоносными зонами. Дебиты родников в них достигают 10 – 25 л/с; в стороне от речных долин вне водоносных зон дебиты не превышают 3 – 5 л/с. Химический состав и минерализация трещинно-карстовых вод изменяются в меридиональном направлении: до широты г. Алапаевск преобладают гидрокарбонатные кальциевые, реже кальциево-магниевые воды с минерализацией от 0,1 до 0,4 г/дм\textsuperscript{3}; южнее распространены гидрокарбонатные кальциево-магниевые, реже кальциевые воды с минерализацией от 0,2 до 0,6–0,8 г/дм\textsuperscript{3}.

\textit{Водоносная зона трещиноватости палеозойских вулканогенно - осадочных и метаморфических образований} ($an, gsPR_1–PZ$) представлена эффузивами различного состава и их туфами, туфоконгломератами и туфопесчаниками различного состава при участии метаморфических зеленых аповулканогенных и глинисто-кремнистых сланцев, слюдяных кристаллосланцев, прослоями метапесчаников и метаморфизованных известняков. Вулканогенные породы и метаморфиты обладают близкими коллекторскими свойствами; подземные воды в них приурочены к грубообломочным и карбонатным прослоям, а также к зонам трещиноватости тектонического происхождения. 

Фильтрационное поле метаморфических пород расчленяется на серию гидравлически связанных и вытянутых по простиранию блоков шириной от десятков до первых километров, структура которых имеет мозаичный характер, обусловленный сочетанием трещин регионального выветривания, разломов и линейных слабопроницаемых экранов. Водоносная трещиноватая зона прослеживается до глубин от 30 до 100 м, преобладающие значения – до 40 – 60 м.

Наиболее глубоко она проникает в горной (открытой) части Уральского складчатого сооружения; близко от поверхности фиксируется в тектонически ослабленных зонах и в долинах рек. На площади Зауральского пенеплена водоносный горизонт перекрыт глинистой корой выветривания и имеет напорный характер до 10–15 м. Водопритоки в скважины составляют от 0,1 – 0,5 до 2 – 3 л/с; в локальных трещинах – в 5–10 раз выше. Дебиты родников в доли нах рек варьируют от 0,5 до 15 – 20 л/с (средний расход 0,8 л/с).
Высокой водообильностью обладают жильные тела, секущие вулканогенно-осадочные породы: дебиты скважин в них составляют 2–3 л/с. На Березовском золоторудном поле, которое представляет собой огромное скопление жил (площадью около 75 км\textsuperscript{2}), шахтный водоотлив в годы наиболее интенсивных горных работ составлял 500 – 600 м\textsuperscript{3}/ч. Питание подземных вод происходит за счет атмосферных осадков; по химическому составу воды гидрокарбонатные кальциево-магниевые, с минерализацией 0,2 – 0,6 г/дм\textsuperscript{3}.

\textit{Водоносная зона трещиноватости позднепалеозойских интрузивных кислых и средних пород} ($\gamma PZ_3$) связана с массивами гранитов, гранодиоритов, плагиогранитов, диоритов. Региональная зона выветривания на этих массивах не превышает 15 – 20 м, зеркало подземных вод в сглаженной форме повторяет современный рельеф. Водоносность зоны крайне неравномерна: в центральных частях массивы практически безводны; по периферии (в приконтактовых частях с другими породами) водоносность возрастает до 0,2 – 0,3 л/с. В частности в тектонических и приконтактовых зонах позднепалеозойского Верхисетского гранитного массива отдельные скважины имели дебит до 7,0 л/с при удельном дебите 2,4 л/с, в центре массива оставаясь безводными. Минерализация вод – в пределах 0,08–0,5 г/дм\textsuperscript{3}; по составу преобладают гидрокарбонатные  кальциево-магниевые.

Гранодиориты и плагиограниты становятся более водоносными за счет секущих позднепалеозойских даек и жил «обновленных», к тому же тектоническими движениями создавшими условия для локализации трещинно-жильных вод в зоне выветривания. В окраинных частях Сысертского массива гранодиоритов и плагиогранитов дебиты скважин изменяются от 1,5 до 10,0 л/с при удельных дебитах 0,1 – 8,0 л/с. В центральной части (где нет жильных тел) в маломощной зоне выветривания дебиты скважин не превышают 0,3 – 0,5 л/с при удельных дебитах 0,001–0,01 л/с. Ближе к периферии фиксируютя мелкие источники с расходом 0,1 – 0,2 л/с. По минерализации и химическому составу воды аналогичны развитым в гранитных массивах. 

\textit{Водоносная зона трещиноватости нижнепалеозойских ультраосновных пород} ($\Sigma PZ_1$)
связана с перидотитами, дунитами и полнопроявленными серпентинитами, образующими в рельефе значительные возвышенности субмеридионального простирания с ограниченными бассейнами питания трещинных грунтовых вод. Породы весьма устойчивы к процессам выветривания; мощности трещиноватой зоны не превышают 10 – 15 м. В центральных частях массивы практически безводны, а в окраинных расход родников варьирует от 0,01 до 0,2 – 0,3 л/с. Дебиты скважин, вскрывших выветрелые трещиноватые серпентиниты, не превышают 1,5 – 2,5 л/с.

Наибольшая обводненность приурочена к периферийным разломам, обновленным неотектоникой. В частности, по отдельным данным, восточная часть Колинского серпентинитового массива (в районе г. Серов) «срезана» в плане и опущена на 200 м региональным ступенчатым сбросом, прослеживаемым на 150 км. Вдоль этого нарушения ультрамафиты трещиноваты и аккумулируют подземный сток грунтовых вод зоны выветривания. Дебиты
скважин в этой зоне изменяются от 5 до 30 л/с при удельных дебитах 0,5 – 6,5 л/с. По периферии других массивов водоносность нередко связана с жилами и дайками кислого и основного составов, имеющими более молодой возраст. Грунтово-трещинные и трещинно-жильные воды имеют минерализацию 0,1–0,5 г/дм\textsuperscript{3}, и лишь на отдельных массивах встречаются ультрапресные воды. По химическому составу они гидрокарбонатные магниевые или гидрокарбонатные магниево-кальциевые. Высокие показатели магния обусловлены большим содержанием его окиси в коренных породах.

Водоносные горизонты зон высокой проницаемости и водоотдачи приурочены к омоложенным в новейшее время дизъюнктивам. Детально одна из таких тектонических зон была изучена А. В. Скалиным в Екатеринбурге
при инженерно-геологических изысканиях под высотные здания в центре города. Она приурочена к контакту габбрового массива с вулканогенной толщей и имеет ширину до 60 м. Коэффициент фильтрации здесь составляет 10 – 20 м/сут (трещиноватое габбро – до 1 м/сут). При опытных откачках дебит скважин в трещиноватых габбро не превышает 1,4 дм\textsuperscript{3}/с, в тектонической зоне – 3,5 дм\textsuperscript{3}/с. Суммарный дебит последней по кустовой откачке трех скважин составил 950 м\textsuperscript{3}/сут.

В целом питание подземных вод Большеуральского бассейна сезонное, за счет инфильтрации атмосферных осадков в теплый период года. Зеркало его вод в сглаженной форме повторяет основные элементы рельефа. На
склонах и уплощенных водоразделах уровни воды залегают на глубинах 5 – 20 м, на хребтах и локальных возвышенностях – до 30 – 50 м. Сравнительно глубокая расчлененность дневной поверхности, особенно в районах приподнятых горных массивов, обеспечивает хорошие условия дренирования водоносных зон речной сетью. Разгрузка вод идет преимущественно вдоль долин рек, а также может быть приурочена к локальным трещинным зонам. Дебиты родников, в зависимости от величины водосборной площади, варьируют от долей до многих десятков литров в секунду.

Эксплуатационные ресурсы Большеуральского бассейна связываются преимущественно с крупными карбонатными массивами среднего и верхнего палеозоя и тектонически активными зонами разломов, на которых возможна организация водозаборов с дебитом 100 – 1000 л/с. На остальных водоносных зонах трещиноватости возможен каптаж подземных вод по отдельным кустам скважин с дебитами от 10 до 30 л/с. Ресурсы порово-пла
стовых вод Иртыш-Обского бассейна связаны преимущественно с опоковым горизонтом палеоцена, мощность обводненной толщи которого 3 – 4 м на западе и до 50 м на востоке. На придолинных участках некоторые водозаборы из него имеют производительность до 50–100 л/с.

Несмотря на значительное количество водоносных горизонтов на изученной площади, наиболее промышленные и обжитые районы Урала испытывают недостаток в водообеспеченности из-за неравномерного распределения ресурсов, а также в связи с невозможностью создания крупных водозаборов с высокой производительностью на локальных участках водоносных зон.


\end{document} % конец документа

