% Этот шаблон документа разработан в 2017 году
% Владимиром Коротковым (kvamob@mail.ru) 
%  

\documentclass[a4paper,12pt]{article} % добавить leqno в [] для нумерации слева

%% Глобальные Параметры страницы
\usepackage[left=3cm,right=2cm,top=1cm,bottom=2cm,bindingoffset=0cm]{geometry}

%\includeonly{O-41/h_granite.tex}

%\usepackage{fp} 						% Вычисления с плавающей точкой
%\usepackage{siunitx}					% При использовании пакета fp все числа должны иметь decimal delimiter точку
%\sisetup{output-decimal-marker={,}}	% Числа выводятся с запятой в качестве разделителя разрядов: \num{3.2} выводит 3,2 

%%% Работа с русским языком
\usepackage{cmap}					% поиск в PDF
\usepackage{mathtext} 				% русские буквы в формулах
\usepackage[T2A]{fontenc}			% кодировка
\usepackage[utf8]{inputenc}			% кодировка исходного текста
\usepackage[english,russian]{babel}	% локализация и переносы

%%% 
%\usepackage{rotating}				% Поворот текста

%%% Дополнительная работа с математикой
\usepackage{amsmath,amsfonts,amssymb,amsthm,mathtools} % AMS
\usepackage{icomma} % "Умная" запятая: $0,2$ --- число, $0, 2$ --- перечисление
\usepackage{gensymb}	% Градусы
%% Номера формул
%\mathtoolsset{showonlyrefs=true} % Показывать номера только у тех формул, на которые есть \eqref{} в  тексте.

%% Шрифты
\usepackage{euscript}	 % Шрифт Евклид
\usepackage{mathrsfs} % Красивый матшрифт


%% Перенос знаков в формулах (по Львовскому)
% \newcommand*{\hm}[1]{#1\nobreak\discretionary{}
%	{\hbox{$\mathsurround=0pt #1$}}{}}

%%% Работа с картинками
\usepackage{graphicx}  % Для вставки рисунков
%\usepackage[export]{adjustbox}
\graphicspath{{images/}}  % папки с картинками
\setlength\fboxsep{3pt} % Отступ рамки \fbox{} от рисунка
\setlength\fboxrule{0.2pt} % Толщина линий рамки \fbox{}
\usepackage{wrapfig} % Обтекание рисунков и таблиц текстом


%%% Работа с таблицами
\usepackage{array,tabularx,tabulary,booktabs} % Дополнительная работа с таблицами
\usepackage{longtable}  % Длинные таблицы
\usepackage{multirow} % Слияние строк в таблице

%%% Подписи к рисункам и таблицам в русской типографской традиции
\usepackage{caption} 
\DeclareCaptionFormat{GOSTtable}{#2#1\\#3}
\DeclareCaptionLabelSeparator{fill}{\hfill}
\DeclareCaptionLabelSeparator{dot}{. }
\DeclareCaptionLabelFormat{fullparents}{\bothIfFirst{#1}{~}#2}
\captionsetup[table]{
	format=GOSTtable,
	font={footnotesize},
	labelformat=fullparents,
	labelsep=fill,
	labelfont=rm,
%	labelfont=it,
	textfont=bf,
	justification=centering,
	singlelinecheck=false
}
\captionsetup{font=small}
\captionsetup[figure]{
	labelsep=dot, 
%	textfont=it
}
% А можно и так
%\captionsetup{labelsep=period}


%%% Модификация команд, задающих разделы
% Не подавлять отступы у первого абзаца

\makeatletter   % Команда \makeatletter делает символ @ буквой, команда \makeatother возвращает всё на свои места.
% Разрешим отступ у первого абзаца
\renewcommand\section{\@startsection {section}{1}{\parindent}%
	{3.5ex \@plus 1ex \@minus .2ex}{2.3ex \@plus.2ex}%
	{\normalfont\hyphenpenalty=10000\Large\bfseries}}

\renewcommand\subsection{\@startsection {subsection}{1}{\parindent}%
	{3.5ex \@plus 1ex \@minus .2ex}{2.3ex \@plus.2ex}%
	{\normalfont\hyphenpenalty=10000\large\bfseries}}
\makeatother

% После номеров разделов \section ставить точки
\usepackage{secdot}			
% И после \subsection тоже ставить точки
\sectiondot{subsection}		

\usepackage{lastpage} % Узнать, сколько всего страниц в документе.

\usepackage{soulutf8} % Модификаторы начертания

%\usepackage{hyperref}
%\usepackage[usenames,dvipsnames,svgnames,table,rgb]{xcolor}
%\hypersetup{				% Гиперссылки
%	unicode=true,           % русские буквы в раздела PDF
%	pdftitle={Заголовок},   % Заголовок
%	pdfauthor={Автор},      % Автор
%	pdfsubject={Тема},      % Тема
%	pdfcreator={Создатель}, % Создатель
%	pdfproducer={Производитель}, % Производитель
%	pdfkeywords={keyword1} {key2} {key3}, % Ключевые слова
%	colorlinks=true,       	% false: ссылки в рамках; true: цветные ссылки
%	linkcolor=red,          % внутренние ссылки
%	citecolor=green,        % на библиографию
%	filecolor=magenta,      % на файлы
%	urlcolor=cyan           % на URL
%}

\usepackage{multicol} % Несколько колонок

\usepackage{fancyhdr} % Колонтитулы
%\pagestyle{fancy}
%\renewcommand{\headrulewidth}{0mm}  % Толщина линейки, отчеркивающей верхний колонтитул
%\lfoot{Нижний левый}
%\rfoot{Нижний правый}
%\rhead{Верхний правый}
%\chead{Верхний в центре}
%\lhead{Верхний левый}
% \cfoot{Нижний в центре} % По умолчанию здесь номер страницы


%%%%%%%%%%%%%%%%%%%%%%%%%%%%%%%%%%%%%%%%%%%%%%%%%%%%%%%%%%%%%%%%%%%%%%%%%%%%%%%%%%%%%%%%%%%%%%%%%%%%%%%%%
%%% PAYLOAD
%%%%%%%%%%%%%%%%%%%%%%%%%%%%%%%%%%%%%%%%%%%%%%%%%%%%%%%%%%%%%%%%%%%%%%%%%%%%%%%%%%%%%%%%%%%%%%%%%%%%%%%%%

%%% Заголовок
\author{ООО <<Гидросфера>>}\label{company}
\title{ОТЧЕТ ПО РЕЗУЛЬТАТАМ ПОИСКОВ ИСТОЧНИКОВ ПОДЗЕМНЫХ ВОД}
\date{\today}
%%%======================================================================================================
\newcommand{\txtExecutor}{ООО <<Гидросфера>>}	% Исполнитель
\newcommand{\txtYear}{2017}						% Год
\newcommand{\txtAddress}{--Address--}			% Адрес
\newcommand{\txtCadaster}{--Cadaster--} 		% Кадастровый номер
%% ГИДРОГЕОЛОГИЧЕСКИЕ УСЛОВИЯ - Раскомментарить одно из:
%\newcommand{\hydrogeology}{\textbf{Трещинные воды}. Уральская сложная гидрогеологическая складчатая область располагается в пределах орографически выраженного одноименного горно - складчатого сооружения. Основными коллекторами подземных вод являются трещиноватые породы коренного субстрата. Мощность зоны региональной трещиноватости составляет в среднем 30  --  100~м. Минимальные значения (20  --  30~м) присущи корам выветривания интрузивных пород; максимальные (60  --  100~м)  ---  карбонатным породам; средние (40  --  60~м)  ---  эффузивно-осадочным и метаморфическим комплексам. Помимо трещин выветривания широко развиты локальные линейные трещинные зоны высокой проницаемости и водоотдачи, связанные с проявлениями дизъюнктивной тектоники и контактами разнородных пород. Подземные воды региональной трещиноватости обычно гидравлически взаимосвязаны, имеют безнапорный характер и образуют небольшие бассейны с интенсивным
водообменном. В вертикальном разрезе фильтрационные свойства пород зоны выветривания неоднородны. По характеру их изменения зона разделяется на три части. В верхней (10  --  20~м), где широко представлены глины или суглинки элювиальной коры выветривания, водопроницаемые свойства очень низки; особенно широко коры распространены на площади Зауральского пенеплена. Средняя часть эрозионной зоны отличается наибольшей активностью, сильной степенью трещиноватости и высокой пористостью (от 1 до 7 \%). В нижней части размеры трещин весьма незначительны, и водоотдача пород практически отсутствует.

\textit{Водоносная зона трещиноватости позднепалеозойских интрузивных кислых и средних пород} ($\gamma PZ_3$) связана с массивами гранитов, гранодиоритов, плагиогранитов, диоритов. Региональная зона выветривания на этих массивах не превышает 15  --  20~м, зеркало подземных вод в сглаженной форме повторяет современный рельеф. Водоносность зоны крайне неравномерна: в центральных частях массивы практически безводны; по периферии (в приконтактовых частях с другими породами) водоносность возрастает до 0,2  --  0,3~л/с. В частности в тектонических и приконтактовых зонах позднепалеозойского Верхисетского гранитного массива отдельные скважины имели дебит до 7,0~л/с при удельном дебите 2,4~л/с, в центре массива оставаясь безводными. Минерализация вод  ---  в пределах 0,08 -- 0,5 г/дм\textsuperscript{3}; по составу преобладают гидрокарбонатные  кальциево-магниевые.

Гранодиориты и плагиограниты становятся более водоносными за счет секущих позднепалеозойских даек и жил «обновленных», к тому же тектоническими движениями создавшими условия для локализации трещинно-жильных вод в зоне выветривания. В окраинных частях Сысертского массива гранодиоритов и плагиогранитов дебиты скважин изменяются от 1,5 до 10,0~л/с при удельных дебитах 0,1  --  8,0~л/с. В центральной части (где нет жильных тел) в маломощной зоне выветривания дебиты скважин не превышают 0,3  --  0,5~л/с при удельных дебитах 0,001 -- 0,01~л/с. Ближе к периферии фиксируютя мелкие источники с расходом 0,1  --  0,2~л/с. По минерализации и химическому составу воды аналогичны развитым в гранитных массивах. 

В целом питание подземных вод Большеуральского бассейна сезонное, за счет инфильтрации атмосферных осадков в теплый период года. Зеркало его вод в сглаженной форме повторяет основные элементы рельефа. На склонах и уплощенных водоразделах уровни воды залегают на глубинах 5  --  20~м, на хребтах и локальных возвышенностях  ---  до 30  --  50~м. Сравнительно глубокая расчлененность дневной поверхности, особенно в районах приподнятых горных массивов, обеспечивает хорошие условия дренирования водоносных зон речной сетью. Разгрузка вод идет преимущественно вдоль долин рек, а также может быть приурочена к локальным трещинным зонам. Дебиты родников, в зависимости от величины водосборной площади, варьируют от долей до многих десятков литров в секунду.

Эксплуатационные ресурсы Большеуральского бассейна связываются преимущественно с крупными карбонатными массивами среднего и верхнего палеозоя и тектонически активными зонами разломов, на которых возможна организация водозаборов с дебитом 100  --  1000~л/с. На остальных водоносных зонах трещиноватости возможен каптаж подземных вод по отдельным кустам скважин с дебитами от 10 до 30~л/с. Ресурсы порово-пластовых вод Иртыш-Обского бассейна связаны преимущественно с опоковым горизонтом палеоцена, мощность обводненной толщи которого 3  --  4~м на западе и до 50~м на востоке. На придолинных участках некоторые водозаборы из него имеют производительность до 50 -- 100~л/с.

Несмотря на значительное количество водоносных горизонтов на изученной площади, наиболее промышленные и обжитые районы Урала испытывают недостаток в водообеспеченности из-за неравномерного распределения ресурсов, а также в связи с невозможностью создания крупных водозаборов с высокой производительностью на локальных участках водоносных зон.
}			% Гранитные интрузии
%\newcommand{\hydrogeology}{\textbf{Трещинные воды}. Уральская сложная гидрогеологическая складчатая область располагается в~пределах орографически выраженного одноименного горно - складчатого сооружения. Основными коллекторами подземных вод являются трещиноватые породы коренного субстрата. Мощность зоны региональной трещиноватости составляет в~среднем 30  --  100~м. Минимальные значения (20  --  30~м) присущи корам выветривания интрузивных пород; максимальные (60  --  100~м)  --  карбонатным породам; средние (40  --  60~м)  ---  эффузивно-осадочным и~метаморфическим комплексам. Помимо трещин выветривания широко развиты локальные линейные трещинные зоны высокой проницаемости и~водоотдачи, связанные с проявлениями дизъюнктивной тектоники и~контактами разнородных пород. Подземные воды региональной трещиноватости обычно гидравлически взаимосвязаны, имеют безнапорный характер и~образуют небольшие бассейны с интенсивным
водообменном. В вертикальном разрезе фильтрационные свойства пород зоны выветривания неоднородны. По характеру их изменения зона разделяется на~три части. В верхней (10  --  20~м), где широко представлены глины или суглинки элювиальной коры выветривания, водопроницаемые свойства очень низки; особенно широко коры распространены на~площади Зауральского пенеплена. Средняя часть эрозионной зоны отличается наибольшей активностью, сильной степенью трещиноватости и~высокой пористостью (от 1 до~7~\%). В нижней части размеры трещин весьма незначительны, и~водоотдача пород практически отсутствует.

\textit{Водоносная зона трещиноватости средне-верхнепалеозойских карбонатных образований} ($сPZ_{2–3}$) является одной из~наиболее водообильных. Она развита преимущественно в~пределах синклинальных, реже антиклинальных структурных форм в~виде отдельных разобщенных либо взаимосвязанных между собой водоносных горизонтов меридионального и~субмеридионального простирания. Наиболее изученные Невьянская, Алапаевская,
Каменская зоны имеют площади до~200 км\textsuperscript{2}. Водовмещающими породами являются известняки с пачками и~прослоями глинистых сланцев, аргиллитов, песчаников и~туффитов. Трещиновато-карстовая зона имеет мощность 50  --  80~м, достигая в~зонах тектонических нарушений 200  --  250~м. С поверхности карбонатные породы осложнены проявлениями карста (в виде воронок); в~пределах мезозойских депрессий карст перекрыт кайнозойскимиотложениями, мощность до~20  --  30~м.

Карстовым процессам подвержены все карбонатные <<массивы>>, но степень их проявления неравномерна. На Алапаевском «массиве» скважинами выявлены погребенные щелевидные депрессии длиной до~900  --  2000~м
при ширине 400  --  500~м и~глубиной от 70 до~140~м в~осевой части.

Характерной особенностью древних карстовых депрессий является высокая трещиноватость бортовых частей и~слабая водонасыщенность днищ. Погребенные карстово - трещинные воды в~депрессиях обладают напором, соответствующим мощности экранирующего покрова. Питание подземных вод осуществляется за счет инфильтрации атмосферных осадков и~разгрузки сопряженных вод из~других горизонтов. Циркуляция происходит по~сложному лабиринту карстовых пустот и~трещин, коэффициент фильтрации в~которых варьирует от 2  --  5 до~30~м/сут. Максимальная водообильность приурочена к~придолинным участкам пересечения с линейными водоносными зонами. Дебиты родников в~них достигают 10  --  25~л/с; в~стороне от речных долин вне водоносных зон дебиты не превышают 3  --  5~л/с. Химический состав и~минерализация трещинно-карстовых вод изменяются в~меридиональном направлении: до~широты г. Алапаевск преобладают гидрокарбонатные кальциевые, реже кальциево-магниевые воды с минерализацией от 0,1 до~0,4 г/дм\textsuperscript{3}; южнее распространены гидрокарбонатные кальциево-магниевые, реже кальциевые воды с минерализацией от 0,2 до~0,6 -- 0,8 г/дм\textsuperscript{3}.

В целом питание подземных вод Большеуральского бассейна сезонное, за счет инфильтрации атмосферных осадков в~теплый период года. Зеркало его вод в~сглаженной форме повторяет основные элементы рельефа. На склонах и~уплощенных водоразделах уровни воды залегают на~глубинах 5  --  20~м, на~хребтах и~локальных возвышенностях  ---  до~30  --  50~м. Сравнительно глубокая расчлененность дневной поверхности, особенно в~районах приподнятых горных массивов, обеспечивает хорошие условия дренирования водоносных зон речной сетью. Разгрузка вод идет преимущественно вдоль долин рек, а также может быть приурочена к~локальным трещинным зонам. Дебиты родников, в~зависимости от величины водосборной площади, варьируют от долей до~многих десятков литров в~секунду.

Эксплуатационные ресурсы Большеуральского бассейна связываются преимущественно с крупными карбонатными массивами среднего и~верхнего палеозоя и~тектонически активными зонами разломов, на~которых возможна организация водозаборов с дебитом 100  --  1000~л/с. На остальных водоносных зонах трещиноватости возможен каптаж подземных вод по~отдельным кустам скважин с дебитами от 10 до~30~л/с. Ресурсы порово-пластовых вод Иртыш-Обского бассейна связаны преимущественно с опоковым горизонтом палеоцена, мощность обводненной толщи которого 3  --  4~м на~западе и~до~50~м на~востоке. На придолинных участках некоторые водозаборы из~него имеют производительность до~50 -- 100~л/с.

Несмотря на~значительное количество водоносных горизонтов на~изученной площади, наиболее промышленные и~обжитые районы Урала испытывают недостаток в~водообеспеченности из-за неравномерного распределения ресурсов, а также в~связи с невозможностью создания крупных водозаборов с высокой производительностью на~локальных участках водоносных зон.
}		% Карбонтаные образования 
%\newcommand{\hydrogeology}{\textbf{Трещинные воды}. Уральская сложная гидрогеологическая складчатая область располагается в пределах орографически выраженного одноименного горно - складчатого сооружения. Основными коллекторами подземных вод являются трещиноватые породы коренного субстрата. Мощность зоны региональной трещиноватости составляет в среднем 30 – 100 м. Минимальные значения (20 – 30 м) присущи корам выветривания интрузивных пород; максимальные (60 – 100 м) – карбонатным породам; средние (40 – 60 м) – эффузивно-осадочным и метаморфическим комплексам. Помимо трещин выветривания широко развиты локальные линейные трещинные зоны высокой проницаемости и водоотдачи, связанные с проявлениями дизъюнктивной тектоники и контактами разнородных пород. Подземные воды региональной трещиноватости обычно гидравлически взаимосвязаны, имеют безнапорный характер и образуют небольшие бассейны с интенсивным водообменном. В вертикальном разрезе фильтрационные свойства пород зоны выветривания неоднородны. По характеру их изменения зона разделяется на три части. В верхней (10 – 20 м), где широко представлены глины или суглинки элювиальной коры выветривания, водопроницаемые свойства очень низки; особенно широко коры распространены на площади Зауральского пенеплена. Средняя часть эрозионной зоны отличается наибольшей активностью, сильной степенью трещиноватости и высокой пористостью (от 1 до 7 \%). В нижней части размеры трещин весьма незначительны, и водоотдача пород практически отсутствует.

\textit{Водоносная зона трещиноватости нижнепалеозойских ультраосновных пород} ($\Sigma PZ_1$)
связана с перидотитами, дунитами и полнопроявленными серпентинитами, образующими в рельефе значительные возвышенности субмеридионального простирания с ограниченными бассейнами питания трещинных грунтовых вод. Породы весьма устойчивы к процессам выветривания; мощности трещиноватой зоны не превышают 10 – 15 м. В центральных частях массивы практически безводны, а в окраинных расход родников варьирует от 0,01 до 0,2 – 0,3 л/с. Дебиты скважин, вскрывших выветрелые трещиноватые серпентиниты, не превышают 1,5 – 2,5 л/с.

Наибольшая обводненность приурочена к периферийным разломам, обновленным неотектоникой. В частности, по отдельным данным, восточная часть Колинского серпентинитового массива (в районе г. Серов) «срезана» в плане и опущена на 200 м региональным ступенчатым сбросом, прослеживаемым на 150 км. Вдоль этого нарушения ультрамафиты трещиноваты и аккумулируют подземный сток грунтовых вод зоны выветривания. Дебиты
скважин в этой зоне изменяются от 5 до 30 л/с при удельных дебитах 0,5 – 6,5 л/с. По периферии других массивов водоносность нередко связана с жилами и дайками кислого и основного составов, имеющими более молодой возраст. Грунтово-трещинные и трещинно-жильные воды имеют минерализацию 0,1–0,5 г/дм\textsuperscript{3}, и лишь на отдельных массивах встречаются ультрапресные воды. По химическому составу они гидрокарбонатные магниевые или гидрокарбонатные магниево-кальциевые. Высокие показатели магния обусловлены большим содержанием его окиси в коренных породах.

Водоносные горизонты зон высокой проницаемости и водоотдачи приурочены к омоложенным в новейшее время дизъюнктивам. Детально одна из таких тектонических зон была изучена А. В. Скалиным в Екатеринбурге
при инженерно-геологических изысканиях под высотные здания в центре города. Она приурочена к контакту габбрового массива с вулканогенной толщей и имеет ширину до 60 м. Коэффициент фильтрации здесь составляет 10 – 20 м/сут (трещиноватое габбро – до 1 м/сут). При опытных откачках дебит скважин в трещиноватых габбро не превышает 1,4 дм\textsuperscript{3}/с, в тектонической зоне – 3,5 дм\textsuperscript{3}/с. Суммарный дебит последней по кустовой откачке трех скважин составил 950 м\textsuperscript{3}/сут.

В целом питание подземных вод Большеуральского бассейна сезонное, за счет инфильтрации атмосферных осадков в теплый период года. Зеркало его вод в сглаженной форме повторяет основные элементы рельефа. На склонах и уплощенных водоразделах уровни воды залегают на глубинах 5 – 20 м, на хребтах и локальных возвышенностях – до 30 – 50 м. Сравнительно глубокая расчлененность дневной поверхности, особенно в районах приподнятых горных массивов, обеспечивает хорошие условия дренирования водоносных зон речной сетью. Разгрузка вод идет преимущественно вдоль долин рек, а также может быть приурочена к локальным трещинным зонам. Дебиты родников, в зависимости от величины водосборной площади, варьируют от долей до многих десятков литров в секунду.

Эксплуатационные ресурсы Большеуральского бассейна связываются преимущественно с крупными карбонатными массивами среднего и верхнего палеозоя и тектонически активными зонами разломов, на которых возможна организация водозаборов с дебитом 100 – 1000 л/с. На остальных водоносных зонах трещиноватости возможен каптаж подземных вод по отдельным кустам скважин с дебитами от 10 до 30 л/с. Ресурсы порово-пластовых вод Иртыш-Обского бассейна связаны преимущественно с опоковым горизонтом палеоцена, мощность обводненной толщи которого 3 – 4 м на западе и до 50 м на востоке. На придолинных участках некоторые водозаборы из него имеют производительность до 50–100 л/с.

Несмотря на значительное количество водоносных горизонтов на изученной площади, наиболее промышленные и обжитые районы Урала испытывают недостаток в водообеспеченности из-за неравномерного распределения ресурсов, а также в связи с невозможностью создания крупных водозаборов с высокой производительностью на локальных участках водоносных зон.
}	% Гипербазиты, серпентиниты и габбро
\newcommand{\hydrogeology}{\textbf{Трещинные воды}. Уральская сложная гидрогеологическая складчатая область располагается в~пределах орографически выраженного одноименного горно -- складчатого сооружения. Основными коллекторами подземных вод являются трещиноватые породы коренного субстрата. Мощность зоны региональной трещиноватости составляет в~среднем 30 –- 100~м. Минимальные значения (20 -- 30~м) присущи корам выветривания интрузивных пород; максимальные (60  -- 100~м)  --- карбонатным породам; средние (40 -- 60~м) --- эффузивно-осадочным и~метаморфическим комплексам. Помимо трещин выветривания широко развиты локальные линейные трещинные зоны высокой проницаемости и~водоотдачи, связанные с проявлениями дизъюнктивной тектоники и~контактами разнородных пород. Подземные воды региональной трещиноватости обычно гидравлически взаимосвязаны, имеют безнапорный характер и~образуют небольшие бассейны с интенсивным водообменном. В вертикальном разрезе фильтрационные свойства пород зоны выветривания неоднородны. По характеру их изменения зона разделяется на~три части. В верхней (10 -- 20~м), где широко представлены глины или суглинки элювиальной коры выветривания, водопроницаемые свойства очень низки; особенно широко коры распространены на~площади Зауральского пенеплена. Средняя часть эрозионной зоны отличается наибольшей активностью, сильной степенью трещиноватости и~высокой пористостью (от 1 до~7~\%). В нижней части размеры трещин весьма незначительны, и~водоотдача пород практически отсутствует.

\textit{Водоносная зона трещиноватости палеозойских вулканогенно -- осадочных и~метаморфических образований} ($an, gsPR_1–PZ$) представлена эффузивами различного состава и~их туфами, туфоконгломератами и~туфопесчаниками различного состава при участии метаморфических зеленых аповулканогенных и~глинисто-кремнистых сланцев, слюдяных кристаллосланцев, прослоями метапесчаников и~метаморфизованных известняков. Вулканогенные породы и~метаморфиты обладают близкими коллекторскими свойствами; подземные воды в~них приурочены к~грубообломочным и~карбонатным прослоям, а также к~зонам трещиноватости тектонического происхождения. 

Фильтрационное поле метаморфических пород расчленяется на~серию гидравлически связанных и~вытянутых по~простиранию блоков шириной от десятков до~первых километров, структура которых имеет мозаичный характер, обусловленный сочетанием трещин регионального выветривания, разломов и~линейных слабопроницаемых экранов. Водоносная трещиноватая зона прослеживается до~глубин от 30 до~100~м, преобладающие значения --- до~40 -- 60~м.

Наиболее глубоко она проникает в~горной (открытой) части Уральского складчатого сооружения; близко от поверхности фиксируется в~тектонически ослабленных зонах и~в~долинах рек. На площади Зауральского пенеплена водоносный горизонт перекрыт глинистой корой выветривания и~имеет напорный характер до~10 -- 15~м. Водопритоки в~скважины составляют от 0,1  -- 0,5 до~2  -- 3~л/с; в~локальных трещинах  --- в~5–10 раз выше. Дебиты родников в~доли нах рек варьируют от 0,5 до~15  -- 20~л/с (средний расход 0,8~л/с).
Высокой водообильностью обладают жильные тела, секущие вулканогенно-осадочные породы: дебиты скважин в~них составляют 2–3~л/с. На Березовском золоторудном поле, которое представляет собой огромное скопление жил (площадью около 75~км\textsuperscript{2}), шахтный водоотлив в~годы наиболее интенсивных горных работ составлял 500  -- 600~м\textsuperscript{3}/ч. Питание подземных вод происходит за счет атмосферных осадков; по~химическому составу воды гидрокарбонатные кальциево-магниевые, с минерализацией 0,2  -- 0,6~г/дм\textsuperscript{3}.

В целом питание подземных вод Большеуральского бассейна сезонное, за счет инфильтрации атмосферных осадков в~теплый период года. Зеркало его вод в~сглаженной форме повторяет основные элементы рельефа. На склонах и~уплощенных водоразделах уровни воды залегают на~глубинах 5  -- 20~м, на~хребтах и~локальных возвышенностях  --- до~30  -- 50~м. Сравнительно глубокая расчлененность дневной поверхности, особенно в~районах приподнятых горных массивов, обеспечивает хорошие условия дренирования водоносных зон речной сетью. Разгрузка вод идет преимущественно вдоль долин рек, а также может быть приурочена к~локальным трещинным зонам. Дебиты родников, в~зависимости от величины водосборной площади, варьируют от долей до~многих десятков литров в~секунду.

Эксплуатационные ресурсы Большеуральского бассейна связываются преимущественно с крупными карбонатными массивами среднего и~верхнего палеозоя и~тектонически активными зонами разломов, на~которых возможна организация водозаборов с дебитом 100 -- 1000~л/с. На остальных водоносных зонах трещиноватости возможен каптаж подземных вод по~отдельным кустам скважин с дебитами от 10 до~30~л/с. Ресурсы порово--пластовых вод Иртыш--Обского бассейна связаны преимущественно с опоковым горизонтом палеоцена, мощность обводненной толщи которого 3 -- 4~м на~западе и~до~50~м на~востоке. На придолинных участках некоторые водозаборы из~него имеют производительность до~50 -- 100~л/с.

Несмотря на~значительное количество водоносных горизонтов на~изученной площади, наиболее промышленные и~обжитые районы Урала испытывают недостаток в~водообеспеченности из-за неравномерного распределения ресурсов, а также в~связи с невозможностью создания крупных водозаборов с высокой производительностью на~локальных участках водоносных зон.
}		% Вулканогенно-осад. и метаморф.
%%
% Геологические условия
\def\txtGeology{
В геологическом  отношении площадка  расположена  \textit{в зоне развития зеленых сланцев и рассланцованных эффузивов}. Скальные породы в верхней зоне сильно выветрелые, разбиты трещинами и  часто оталькованы.
}
% Результаты рекогносцировочных работ
\def\txtRecog{
Участок работ сложен метаморфизованными сланцами, с включениями оталькованных участков. Верхняя часть разреза сложена слабопроницаемыми грунтами, поэтому существует реальная угроза подтопления зданий и сооружений и заболачивания участка. На участке имеются выходы подземных выработок.
}
% Камеральные работы
\def\txtCamer{
\begin{itemize}
	\item В пределах перспективного участка выделена серия водоносных зон сосредоточенной разгрузки подземных вод.
	\item Наличие трещиноватых зон на глубине от 45 до 70 м в месте сосредоточенной разгрузки подземных вод.
	\item Мощность выявленных водоносных зоны 0,6 м, простирание – до 1,5 – 2,0  км.
	\item Место заложения скважины водоснабжения рекомендуется выбирать \textbf{в пределах узлов пересечения выявленных водоносных зон}.
\end{itemize}
}
% Рекомендации
\def\txtRecommend{
\begin{itemize}
	\item Для обеспечения требуемых потребностей в воде рекомендуется проходка скважины водоснабжения в пределах узлов пересечения выявленных водоносных зон сосредоточенной разгрузки подземных вод. Ориентировочная глубина скважины в этом случае составит 40 – 45 м. Если расположить скважину вне пределов этих узлов, ее глубина составит не менее 70 м.
	\item Положение скважины будет помечено на местности репером.
	\item Обсадку трубами беcщелевой перфорации целесообразно выполнить  до коренных грунтов на глубину до 30  м от поверхности земли.
	\item Во избежание попадания верховодки целесообразно применять обсадку трубами до глубины 5 - 10 м. 
	\item Крайне не рекомендуется устройство заглублений вокруг устья скважины, для отвода воды из скважины необходимо использовать \textbf{водопровод  незаглубленного типа}, так как водопроводная траншея по сути является дренажной канавой, собирающей в себя поверхностные воды и «верховодку» далеко не лучшего качества, которые к тому же вымывают песок из суглинков, далее эти загрязненные воды с песком по затрубному пространству поступают в скважину, вызывая загрязнение и запесочивание скважины.
\end{itemize}
}


%%%%%%%%%%%%%%%%%%%%%%%%%%%%%%%%%%%%%%%%%%%%%%%%%%%%%%%%%%%%%%%%%%%%%%%%%%%%%%%%%%%%%%%%%%%%%%%%%%%%%%%%%

\begin{document} % конец преамбулы, начало документа

\setlength{\extrarowheight}{1mm} % Дополнительный интервал между строками таблиц

%% Титульная страница

\begin{titlepage}
	\begin{center}
		\textbf{\txtExecutor}
		\vspace{7.5cm}
		
		{\LARGE ОТЧЕТ ПО РЕЗУЛЬТАТАМ ПОИСКОВ}

		\bigskip

		{\LARGE ИСТОЧНИКОВ ПОДЗЕМНЫХ ВОД}
		
		\bigskip
		
		на участке по адресу:
				
		\underline{\txtAddress}
		
		\bigskip
		Кадастровый номер \txtCadaster
		
		\vfill
	
		\bigskip
		
	\end{center}

	\vfill
	
	\newlength{\ML}
	\settowidth{\ML}{«\underline{\hspace{0.7cm}}» \underline{\hspace{2cm}}}
	\hfill
	\begin{minipage}{1.0\textwidth}
		Директор ООО <<Гидросфера>> к.г.м.н.
		\underline{\hspace{\ML}} А.\,А.~Кашкаров\\
	\end{minipage}%
	
	\bigskip
	
	\vfill
	\begin{center}
		Екатеринбург, \txtYear
	\end{center}			

	\end{titlepage}

%%%%%%%%%%%%%%%%%%%%%%%%%%%%%%%%%%%%%%%%%%%%%%%%%%%%%%%%%%%%%%%%%%%%%%%%%%%%%%%%

\section*{Введение}

Изыскания источников подземных вод – важнейший этап в производстве комплексных работ по водоснабжению землепользователей за счет источников подземных вод, основанный изучении миграции подземных вод к местам их разгрузки.

Круговорот воды в природе чаще всего воспринимается как сток поверхностных вод к морям и океанам. Однако согласно статистическим сведениям, под землей хранится и перемещается на порядок больший объем воды, чем в реках и ручьях.

Теоретические и практические работы, выполненные автором настоящего заключения, убедительно показали, что участки земной коры, в которых наблюдается эффективная фильтрация подземной воды, имеют в поперечнике размер от 0,5 до 1,0 метра, по протяженности более 2 км, по глубине не превышает высоту изучаемого участка над уровнем моря. Такие участки наиболее перспективны для организации водозаборов. Именно они являются объектом изысканий, поскольку при минимальной глубине из них можно получить максимальный объем воды наиболее высокого качества. Качество воды определяется ее высокой обновляемостью в водоносных зонах, исключающую застойность, высокую минерализацию и жесткость.

Методы изысканий предполагают использование комплекса электроразведочных работ и оригинальный метод интерпретации, связанный с трансформацией геоэлектрических моделей грунтов в гидрогеологические и геомеханические модели.

\section{Цель изыскательских работ}
Целью выполненного комплекса изыскательских работ являются:
\begin{itemize}
	\item изучение геологического строения верхней части земной коры, литологии, стратиграфии процессов выветривания, техногенного изменения, тектоники и т.д.
	\item гидрогеологических условий верхней части земной коры с точки зрения фильтрационных возможностей грунтов, обеспечивающих перенос подземных вод к местам их разгрузки.
\end{itemize}


\section{Объёмы и виды изыскательских работ}
Для расшифровки геологической и инженерно-геологической ситуации на территории изучаемого участка землепользования выполнен следующий объем работ, приведенный в табл. {\ref{t:volumes}}.

\begin{table}\footnotesize
\caption{Объемы и виды выполненных работ}
\label{t:volumes}
\centering
\begin{tabulary}{\textwidth}{|C|L|L|L|L|}
	\hline 
	№№ п/п & Наименование работ & Ед. изм. & Объём & Решаемые задачи \\ 
	\hline 
	1. & Архивные и фондовые работы & печ. стр. & 200 & Оценка состояния свойств и геологии района работ \\ 
	\hline 
	2. & Рекогносцировочные работы & км & 4,0 & Оценка рельефа местности, описание обнажений  и выходов подземных и поверхностных вод \\ 
	\hline 
	3. & Геофизические работы методами электроразведки & точка & 30,0 & Оценка строения земной коры на глубину и по площади \\ 
	\hline 
	4. & Камеральные работы & \% от полевых & 30,0 & Описание и обработка материалов изысканий. Составление заключения \\ 
	\hline 
\end{tabulary} 
\end{table}

\section{Методика работ}

\subsection{Архивные и фондовые работы}
Архивные и фондовые работы выполняются квалифицированными инженерными работниками геологической специализации и связаны с анализом результатов комплекса геологических и инженерно-геологических работ прошлых лет.

На основании комплекса архивных и фондовых работ при анализе отчетного опубликованного и картографического материала дается характеристика геологии района и участка работ, состава и состояния грунтов по их устойчивости и перспектив водоносности, рельефа местности.

При качественном выполнении архивных и фондовых работ выдаются наиболее точные рекомендации по технологии изыскательских и буровых работ и прогноз объемов, дебита проектируемых скважин и их конструкции.

\subsection{Рекогносцировочные работы}
Рекогносцировочные работы выполняют на изучаемой местности путем ее осмотра квалифицированным горным инженером – геологом методом маршрутных съемок по профилям, ориентированным вкрест выделенных геоморфологических структур или геологических неоднородностей.

При рекогносцировочных работах выполняют описание геоморфологии местности, обнажений, выходов подземных и поверхностных вод.

\subsection{Геофизические работы}
Если при изысканиях не применяют вскрышные работы, то геофизические изыскательские мероприятия являются основными методическими приемами инженерно-геологического и гидрогеологического исследования участка работ.

При изысканиях, связанных с оценкой инженерно-геологических и гидрогеологических параметров горных пород и массивов, связанных с оценкой устойчивости оснований фундаментов инженерных сооружений и оценкой перспективности участка для решения задач водоснабжения за счет источников подземных вод используют электроразведочные методы в глубинном и площадном вариантах.

Чаще всего используют три наиболее эффективных метода:  ВЭЗ (вертикальное электрическое зондирование), МСГ (метод срединных градиентов), ЕП (метод естественного поля).

Методом вертикального электрического зондирования изучается разрез грунтов на глубину и оцениваются геомеханические и фильтрационные параметры грунтов на разных глубинах.

Методом срединных градиентов изучается геологическое, гидрогеологическое и инженерно-геологическое строение земной коры по площади.

Методом естественного поля изучается гидрогеологическое строение земной коры в плане выявления зон, активно фильтрующих воду.

\section{Результаты работ}
\subsection{Результаты архивных и фондовых работ}
Участок работ расположен по адресу: \txtAddress.

\subsection*{Географические сведения}
В пределах листа О-41 выделяются несколько орографических районов (с запада на восток): 1 – область горно-останцового рельефа водораздельной части Среднего Урала с отметками вершин до 609 м (гора Березовая) и депрессий до 300–400 м, развитого на сильно деформированных метаморфических толщах допалеозоя (Уфалейский комплекс), раннего палеозоя и массивах перидотит-габбрового Платиноносного пояса (Ревдинский массив); 2 – область сильно выровненного увалистого рельефа восточного склона Среднего Урала с беспорядочным расположением низких
увалов высотой 300–400 м и единичными вершинами (гора Карабайка – 544 м), сложенная преимущественно ранне- и среднепалеозойскими вулканогенными и интрузивными породами; 3 – пенеплен: таежно-лесистая в северной части и лесостепная в южной, часто заболоченная равнина, сложенная в западной части разнообразными палеозойскими и допалеозойскими породами, в восточной – мезозойскими и кайнозойскими осадками чехла Западно-Сибирской плиты. 
\begin{figure}[h]
	\fbox{\includegraphics[width=\textwidth]{map}}
	%		\fbox{\includegraphics[width=650px]{map.png}}
	\caption{Обзорная карта района работ}
\end{figure}
Наблюдается общее понижение рельефа в северо-восточном направлении, где присутствуют крупные болота и озера. Абсолютные отметки колеблются от 37 м на северо-востоке (оз. Леушинский Туман) до 330 м (гора Известковая) на юго-западе, относительные превышения составляют в среднем 50–100 м (до 50 м в восточной части). Крутизна склонов составляет обычно 6–10\degree, редко до 20\degree.
Климат континентальный, среднегодовая температура около 0 \degreeС. Средняя температура декабря и января от –10 до –17 \degreeС (минимальная –52 \degreeС).

Средняя температура июля от +16 до +17\degree (максимальная +37 \degreeС). Осадки в июле и августе составляют 360–426 мм, а в январе–феврале – 100–200 мм.
Установление снежного покрова наблюдается в ноябре. Почвы, промерзая местами до 1 м, оттаивают в мае. Большая часть площади занята лесами; реликтовые леса состоят в основном из ели, пихты, кедра, сосны и березы, а на старых вырубках и гарях преобладают березово-осиновые. В южной части территории листа О-41 присутствуют фрагменты лесостепных ландшафтов, представленных разнотравьем с осиново-березовыми колками, в северо-восточной части широко представлены залесенные и открытые болота с преобладающим моховым покрытием. Значительная часть южной и западной частей района работ представлена сельскохозяйственными угодьями, занятыми кормовыми, овощными и злаковыми культурами. 

Речная сеть территории относится главным образом к бассейну р. Обь – реки Исеть, Нейва, Реж, Тагил, Тура, Ляля, Лобва, Тавда, Конда, и только несколько водотоков в юго-западной части – к бассейну р. Кама (реки Чусовая и Уфалейка). Речные долины преимущественно широкие, хорошо проработанные, в западной части отмечаются и каньонообразные. Питание рек происходит за счет талых вод и атмосферных осадков, в меньшей степени – за счет подземных вод. Реки на большей части площади не судоходны (за исключением рек Тавда и Тура в нижнем течении), наиболее крупные из них пригодны для прохождения на лодках. Ледостав начинается в начале ноября, ледоход – во второй половине апреля, толщина льда достигает 1 м. Для нужд энергетики и металлургического производства на всех
главных реках западной и южной частей площади созданы пруды.

\begin{figure}[h]
	\fbox{\includegraphics[width=\textwidth]{geomap}}
	%		\fbox{\includegraphics[width=650px]{map.png}}
	\caption{Геологическая карта района работ}
\end{figure}

\subsection*{Геологические условия}
\txtGeology

\subsection*{Коры выветривания}
%Лист O-41
Коры выветривания относятся к линейному и площадному типам. Площадное распространение их определяется геоморфологическим строением, а тип коры – литологическим и петрографическим составом пород субстрата. Наиболее хорошо от последующих размывов коры сохранились в пределах депрессий, где наблюдается полный их профиль, а  мощность кор достигает 40 – 80 м. На пологих склонах водоразделов и холмов мощность кор от первых метров до 15–20 м.

Линейные коры приурочены к зонам контактов и разрывных нарушений. К линейному типу относятся и контактово-карстовые коры, приуроченные к тектоническим контактам силикатных пород и карстующихся
известняков. Линейные коры распространяются на глубину от нескольких десятков до полутора сотен метров при протяженности от нескольких десятков до нескольких сот километров. В разрезе кор выветривания, независимо от состава материнских пород, отчетливо выделяются три зоны, связанные взаимными переходами (снизу вверх): 
\begin{itemize}
\item зона дезинтеграции (зона щебнистых и дресвяно-щебнистых продуктов);
\item промежуточная зона зона глинисто-дресвянистых продуктов;
\item зона конечной гидратации (зона глинистых продуктов).
\end{itemize}


\textit{Зона дезинтеграции} характеризуется начальным выветриванием с образованием щебенистого или щебнисто-дресвянистого элювия и появлением в породах гидрохлорита. 

В \textit{зоне промежуточных продуктов} увеличивается содержание гипергенных минералов: гидрохлорита, гидрослюды и каолинита. 

\textit{Зона конечной гидратации} представлена пестроцветными, буроватыми, желтоватыми, зеленовато-серыми или темно-зелеными, часто структурными, глинами гидро\-слюдисто-каолинитового, гидрослюдисто-монтмориллонитового состава, нередко с галлуазитом, гидрослюдой, гиббситом, иногда алунитом; в корах по гранитоидам преобладают каолинит и кварц, а в корах по ультрамафитам широко развиты нонтронит и флогопит. Минералы тяжелой фракции представлены в основном пиритом, сидеритом, лимонитом, мартитом, лейкоксеном. Вещественный состав кор выветривания обусловлен преобладанием одного из двух минералого-геохимических типов – ферритно-сиаллитного, развитого по породам основного состава, и сиаллитного, характерного для пород среднего и кислого составов.

\subsection*{Четвертичные образования}
%Лист O-41
Четвертичные образования имеют широкое распространение на площади листа. Они представлены генетическими типами: флювиальными – аллювий, лимний, лимноаллювий, палюстрий, лимнопалюстрий; субаэральными – элювий, делювий, элювиоделювий, эолий, лессоиды; ледниковыми – гляциалий, флювиогляциалий, лимногляциалий. В речных долинах превалируют аллювиальные и лимноаллювиальные отложения, мощность которых 10 – 15 м в верхнем течении и 30 – 40 до 60 м – в нижнем. На междуречьях широко распространен лимний в озерных ваннах и понижениях палеорельефа мощностью 5–10 до 35 м, а также образования субаэрального и палюстринного происхождения, мощность которых невелика – 
2–3 до 8 м. Ледниковые образования развиты локально в северо-западном углу планшета и имеют изменчивую мощность от 5–10 до 25 м.
Возраст образований определен на основании биостратиграфических данных с учетом палеомагнитных исследований и радиоуглеродных датировок, а также в соответствии со схемами стратиграфии Урала и утвержденными легендами.

\subsection*{Гидрогеологические условия}
%Лист O-41
%\textbf{Трещинные воды}. Уральская сложная гидрогеологическая складчатая область располагается в пределах орографически выраженного одноименного горно - складчатого сооружения. Основными коллекторами подземных вод являются трещиноватые породы коренного субстрата. Мощность зоны региональной трещиноватости составляет в среднем 30  --  100~м. Минимальные значения (20  --  30~м) присущи корам выветривания интрузивных пород; максимальные (60  --  100~м)  ---  карбонатным породам; средние (40  --  60~м)  ---  эффузивно-осадочным и метаморфическим комплексам. Помимо трещин выветривания широко развиты локальные линейные трещинные зоны высокой проницаемости и водоотдачи, связанные с проявлениями дизъюнктивной тектоники и контактами разнородных пород. Подземные воды региональной трещиноватости обычно гидравлически взаимосвязаны, имеют безнапорный характер и образуют небольшие бассейны с интенсивным
водообменном. В вертикальном разрезе фильтрационные свойства пород зоны выветривания неоднородны. По характеру их изменения зона разделяется на три части. В верхней (10  --  20~м), где широко представлены глины или суглинки элювиальной коры выветривания, водопроницаемые свойства очень низки; особенно широко коры распространены на площади Зауральского пенеплена. Средняя часть эрозионной зоны отличается наибольшей активностью, сильной степенью трещиноватости и высокой пористостью (от 1 до 7 \%). В нижней части размеры трещин весьма незначительны, и водоотдача пород практически отсутствует.

\textit{Водоносная зона трещиноватости позднепалеозойских интрузивных кислых и средних пород} ($\gamma PZ_3$) связана с массивами гранитов, гранодиоритов, плагиогранитов, диоритов. Региональная зона выветривания на этих массивах не превышает 15  --  20~м, зеркало подземных вод в сглаженной форме повторяет современный рельеф. Водоносность зоны крайне неравномерна: в центральных частях массивы практически безводны; по периферии (в приконтактовых частях с другими породами) водоносность возрастает до 0,2  --  0,3~л/с. В частности в тектонических и приконтактовых зонах позднепалеозойского Верхисетского гранитного массива отдельные скважины имели дебит до 7,0~л/с при удельном дебите 2,4~л/с, в центре массива оставаясь безводными. Минерализация вод  ---  в пределах 0,08 -- 0,5 г/дм\textsuperscript{3}; по составу преобладают гидрокарбонатные  кальциево-магниевые.

Гранодиориты и плагиограниты становятся более водоносными за счет секущих позднепалеозойских даек и жил «обновленных», к тому же тектоническими движениями создавшими условия для локализации трещинно-жильных вод в зоне выветривания. В окраинных частях Сысертского массива гранодиоритов и плагиогранитов дебиты скважин изменяются от 1,5 до 10,0~л/с при удельных дебитах 0,1  --  8,0~л/с. В центральной части (где нет жильных тел) в маломощной зоне выветривания дебиты скважин не превышают 0,3  --  0,5~л/с при удельных дебитах 0,001 -- 0,01~л/с. Ближе к периферии фиксируютя мелкие источники с расходом 0,1  --  0,2~л/с. По минерализации и химическому составу воды аналогичны развитым в гранитных массивах. 

В целом питание подземных вод Большеуральского бассейна сезонное, за счет инфильтрации атмосферных осадков в теплый период года. Зеркало его вод в сглаженной форме повторяет основные элементы рельефа. На склонах и уплощенных водоразделах уровни воды залегают на глубинах 5  --  20~м, на хребтах и локальных возвышенностях  ---  до 30  --  50~м. Сравнительно глубокая расчлененность дневной поверхности, особенно в районах приподнятых горных массивов, обеспечивает хорошие условия дренирования водоносных зон речной сетью. Разгрузка вод идет преимущественно вдоль долин рек, а также может быть приурочена к локальным трещинным зонам. Дебиты родников, в зависимости от величины водосборной площади, варьируют от долей до многих десятков литров в секунду.

Эксплуатационные ресурсы Большеуральского бассейна связываются преимущественно с крупными карбонатными массивами среднего и верхнего палеозоя и тектонически активными зонами разломов, на которых возможна организация водозаборов с дебитом 100  --  1000~л/с. На остальных водоносных зонах трещиноватости возможен каптаж подземных вод по отдельным кустам скважин с дебитами от 10 до 30~л/с. Ресурсы порово-пластовых вод Иртыш-Обского бассейна связаны преимущественно с опоковым горизонтом палеоцена, мощность обводненной толщи которого 3  --  4~м на западе и до 50~м на востоке. На придолинных участках некоторые водозаборы из него имеют производительность до 50 -- 100~л/с.

Несмотря на значительное количество водоносных горизонтов на изученной площади, наиболее промышленные и обжитые районы Урала испытывают недостаток в водообеспеченности из-за неравномерного распределения ресурсов, а также в связи с невозможностью создания крупных водозаборов с высокой производительностью на локальных участках водоносных зон.

\hydrogeology

\begin{figure}[!h]
	\centering
	\includegraphics[height=0.95\textheight]{Legend-o-41-25}
	\caption[Условные обозначения]{Условные обозначения к геологической карте}
	\label{img:legend}
\end{figure}

\subsection{Результаты рекогносцировочных работ}
Рекогносцировочные работы, выполненные методом маршрутной съемки в районе площадки проектируемого строительства, позволили получить следующие результаты:

\txtRecog

\subsection{Камеральные работы}
На основе выполненных камеральных работ установлено:
\txtCamer

\section{Рекомендации}
\txtRecommend


\end{document} % конец документа

