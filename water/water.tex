% Этот шаблон документа разработан в 2017 году
% Владимиром Коротковым (kvamob@mail.ru) 
%  

\documentclass[a4paper,12pt]{article} % добавить leqno в [] для нумерации слева

%% Глобальные Параметры страницы
%\usepackage[left=3cm,right=2cm,top=1cm,bottom=2cm,bindingoffset=0cm]{geometry}

%\usepackage{fp} 						% Вычисления с плавающей точкой
%\usepackage{siunitx}					% При использовании пакета fp все числа должны иметь decimal delimiter точку
%\sisetup{output-decimal-marker={,}}	% Числа выводятся с запятой в качестве разделителя разрядов: \num{3.2} выводит 3,2 

%%% Работа с русским языком
\usepackage{cmap}					% поиск в PDF
\usepackage{mathtext} 				% русские буквы в формулах
\usepackage[T2A]{fontenc}			% кодировка
\usepackage[utf8]{inputenc}			% кодировка исходного текста
\usepackage[english,russian]{babel}	% локализация и переносы

%%% 
%\usepackage{rotating}				% Поворот текста

%%% Дополнительная работа с математикой
\usepackage{amsmath,amsfonts,amssymb,amsthm,mathtools} % AMS
\usepackage{icomma} % "Умная" запятая: $0,2$ --- число, $0, 2$ --- перечисление

%% Номера формул
%\mathtoolsset{showonlyrefs=true} % Показывать номера только у тех формул, на которые есть \eqref{} в  тексте.

%% Шрифты
\usepackage{euscript}	 % Шрифт Евклид
\usepackage{mathrsfs} % Красивый матшрифт


%% Перенос знаков в формулах (по Львовскому)
% \newcommand*{\hm}[1]{#1\nobreak\discretionary{}
%	{\hbox{$\mathsurround=0pt #1$}}{}}

%%% Работа с картинками
\usepackage{graphicx}  % Для вставки рисунков
%\usepackage[export]{adjustbox}
\graphicspath{images/}  % папки с картинками
\setlength\fboxsep{3pt} % Отступ рамки \fbox{} от рисунка
\setlength\fboxrule{0.2pt} % Толщина линий рамки \fbox{}
\usepackage{wrapfig} % Обтекание рисунков и таблиц текстом


%%% Работа с таблицами
\usepackage{array,tabularx,tabulary,booktabs} % Дополнительная работа с таблицами
\usepackage{longtable}  % Длинные таблицы
\usepackage{multirow} % Слияние строк в таблице

%%% Подписи к рисункам и таблицам в русской типографской традиции
\usepackage{caption} 
\DeclareCaptionFormat{GOSTtable}{#2#1\\#3}
\DeclareCaptionLabelSeparator{fill}{\hfill}
\DeclareCaptionLabelSeparator{dot}{. }
\DeclareCaptionLabelFormat{fullparents}{\bothIfFirst{#1}{~}#2}
\captionsetup[table]{
	format=GOSTtable,
	font={footnotesize},
	labelformat=fullparents,
	labelsep=fill,
	labelfont=rm,
%	labelfont=it,
	textfont=bf,
	justification=centering,
	singlelinecheck=false
}
\captionsetup{font=small}
\captionsetup[figure]{
	labelsep=dot, 
%	textfont=it
}
% А можно и так
%\captionsetup{labelsep=period}


%%% Модификация команд, задающих разделы
% Не подавлять отступы у первого абзаца

\makeatletter   % Команда \makeatletter делает символ @ буквой, команда \makeatother возвращает всё на свои места.
% Разрешим отступ у первого абзаца
\renewcommand\section{\@startsection {section}{1}{\parindent}%
	{3.5ex \@plus 1ex \@minus .2ex}{2.3ex \@plus.2ex}%
	{\normalfont\hyphenpenalty=10000\Large\bfseries}}
\makeatother

% После номеров разделов \section ставить точки
\usepackage{secdot}			
% И после \subsection тоже ставить точки
\sectiondot{subsection}		


%%%%%%%%%%%%%%%%%%%%%%%%%%%%%%%%%%%%%%%%%%%%%%%%%%%%%%%%%%%%%%%%%%%%%%%%%%%%%%%%%%%%%%%%%%%%%%%%%%%%%%%%%
%%% PAYLOAD
%%%%%%%%%%%%%%%%%%%%%%%%%%%%%%%%%%%%%%%%%%%%%%%%%%%%%%%%%%%%%%%%%%%%%%%%%%%%%%%%%%%%%%%%%%%%%%%%%%%%%%%%%

%%% Заголовок
\author{ООО <<Гидросфера>>}\label{company}
\title{ОТЧЕТ ПО РЕЗУЛЬТАТАМ ПОИСКОВ ИСТОЧНИКОВ ПОДЗЕМНЫХ ВОД}
\date{\today}
%%%======================================================================================================
\newcommand{\txtExecutor}{ООО <<Гидросфера>>}	% Исполнитель
\newcommand{\txtYear}{2017}						% Год
\newcommand{\txtAddress}{--Address--}			% Адрес
\newcommand{\txtCadaster}{--Cadaster--} 		% Кадастровый номер


%%%%%%%%%%%%%%%%%%%%%%%%%%%%%%%%%%%%%%%%%%%%%%%%%%%%%%%%%%%%%%%%%%%%%%%%%%%%%%%%%%%%%%%%%%%%%%%%%%%%%%%%%

\begin{document} % конец преамбулы, начало документа

\setlength{\extrarowheight}{1mm} % Дополнительный интервал между строками таблиц

%% Титульная страница

\begin{titlepage}
	\begin{center}
		\textbf{\txtExecutor}
		\vspace{5.5cm}
		
		{\LARGE ОТЧЕТ ПО РЕЗУЛЬТАТАМ ПОИСКОВ}

		\bigskip

		{\LARGE ИСТОЧНИКОВ ПОДЗЕМНЫХ ВОД}
		
		\bigskip
		
		на участке по адресу:
				
		\underline{\txtAddress}
		
		\bigskip
		Кадастровый номер \txtCadaster
		
		\vfill
	
		\bigskip
		
	\end{center}

	\vfill
	
	\newlength{\ML}
	\settowidth{\ML}{«\underline{\hspace{0.7cm}}» \underline{\hspace{2cm}}}
	\hfill
	\begin{minipage}{1.0\textwidth}
		Директор ООО <<Гидросфера>> к.г.м.н.
		\underline{\hspace{\ML}} А.\,А.~Кашкаров\\
	\end{minipage}%
	
	\bigskip
	
	\vfill
	\begin{center}
		Екатеринбург, \txtYear
	\end{center}			

	\end{titlepage}

%%%%%%%%%%%%%%%%%%%%%%%%%%%%%%%%%%%%%%%%%%%%%%%%%%%%%%%%%%%%%%%%%%%%%%%%%%%%%%%%

\section*{Введение}

Изыскания источников подземных вод – важнейший этап в производстве комплексных работ по водоснабжению землепользователей за счет источников подземных вод, основанный изучении миграции подземных вод к местам их разгрузки.

Круговорот воды в природе чаще всего воспринимается как сток поверхностных вод к морям и океанам. Однако согласно статистическим сведениям, под землей хранится и перемещается на порядок больший объем воды, чем в реках и ручьях.

Теоретические и практические работы, выполненные автором настоящего заключения, убедительно показали, что участки земной коры, в которых наблюдается эффективная фильтрация подземной воды, имеют в поперечнике размер от 0,5 до 1,0 метра, по протяженности более 2 км, по глубине не превышает высоту изучаемого участка над уровнем моря. Такие участки наиболее перспективны для организации водозаборов. Именно они являются объектом изысканий, поскольку при минимальной глубине из них можно получить максимальный объем воды наиболее высокого качества. Качество воды определяется ее высокой обновляемостью в водоносных зонах, исключающую застойность, высокую минерализацию и жесткость.

Методы изысканий предполагают использование комплекса электроразведочных работ и оригинальный метод интерпретации, связанный с трансформацией геоэлектрических моделей грунтов в гидрогеологические и геомеханические модели.

\section{Цель изыскательских работ}
Целью выполненного комплекса изыскательских работ являются:
\begin{itemize}
	\item изучение геологического строения верхней части земной коры, литологии, стратиграфии процессов выветривания, техногенного изменения тектоники и т.д.
	\item инженерно-геологического строения верхней части земной коры с точки зрения фильтрационных возможностей грунтов, обеспечивающих перенос подземных вод к местам их разгрузки.
\end{itemize}


\section{Объёмы и виды изыскательских работ}
Для расшифровки геологической и инженерно-геологической ситуации на территории изучаемого участка землепользования выполнен следующий объем работ, приведенный в табл. {\ref{t:volumes}}.

\begin{table}\footnotesize
\caption{Объемы и виды выполненных работ}
\label{t:volumes}
\centering
\begin{tabulary}{\textwidth}{|C|L|L|L|L|}
	\hline 
	№№ п/п & Наименование работ & Ед. изм. & Объём & Решаемые задачи \\ 
	\hline 
	1. & Архивные и фондовые работы & печ. стр. & 200 & Оценка состояния свойств и геологии района работ \\ 
	\hline 
	2. & Рекогносцировочные работы & км & 4,0 & Оценка рельефа местности, описание обнажений  и выходов подземных и поверхностных вод \\ 
	\hline 
	3. & Геофизические работы методами электроразведки & точка & 30,0 & Оценка строения земной коры на глубину и по площади \\ 
	\hline 
	4. & Камеральные работы & \% от полевых & 30,0 & Описание и обработка материалов изысканий. Составление заключения \\ 
	\hline 
\end{tabulary} 
\end{table}

\section{Методика работ}

\subsection{Архивные и фондовые работы}
Архивные и фондовые работы выполняются квалифицированными инженерными работниками геологической специализации и связаны с анализом результатов комплекса геологических и инженерно-геологических работ прошлых лет.
На основании комплекса архивных и фондовых работ при анализе отчетного опубликованного и картографического материала дается характеристика геологии района и участка работ, состава и состояния грунтов по их устойчивости и перспектив водоносности, рельефа местности.
При грамотном выполнении архивных и фондовых работ дают наиболее точные рекомендации по технологии изыскательских и буровых работ и прогноз объемов, дебита проектируемых скважин и их конструкции.

\subsection{Рекогносцировочные работы}
Рекогносцировочные работы выполняют на изучаемой местности путем ее осмотра квалифицированным горным инженером – геологом методом маршрутных съемок по профилям, ориентированным вкрест выделенных геоморфологических структур или геологических неоднородностей.
При рекогносцировочных работах выполняют описание геоморфологии местности, обнажений, выходов подземных и поверхностных вод.

\subsection{Геофизические работы}
Если при изысканиях не применяют вскрышные работы, то геофизические изыскательские мероприятия являются основными методическими приемами инженерно-геологического и гидрогеологического исследования участка работ.
При изысканиях, связанных с оценкой инженерно-геологических и гидрогеологических параметров горных пород и массивов, связанных с оценкой устойчивости оснований фундаментов инженерных сооружений и оценкой перспективности участка для решения задач водоснабжения за счет источников подземных вод, чаще всего используют электроразведочные методы в глубинном и площадном вариантах.
Используют три наиболее эффективных метода:  ВЭЗ (вертикальное электрическое зондирование), МСГ (метод срединных градиентов), ЕП (метод естественного поля).
Методом вертикального электрического зондирования изучается разрез грунтов на глубину и оцениваются геомеханические и фильтрационные параметры грунтов на разных глубинах.
Методом срединных градиентов изучается геологическое, гидрогеологическое и инженерно-геологическое строение земной коры по площади.
Методом естественного поля изучается гидрогеологическое строение земной коры в плане выявления зон, активно фильтрующих воду.




ммммммммммммммммммммммммммммммммммммммммммм
\begin{enumerate}
\item Для сброса дренажных вод на участке по контуру здания бурятся дренажные скважины на глубину 30 м каждая с обсадкой перфорированными полиэтиленовыми трубами.
\item В устье дренажных скважин производится выемка грунта на ширину дренажного колодца в нижней части  с расширением к верхней части. Ширина выемки в верхней части составляет 2 – 2,5 диаметра дренажного колодца. Глубина выемок 2,5 – 2,7 м. В выемках устанавливаются  дренажные колодцы (Приложение 4). Промежуток между внешними стенками дренажных колодцев и стенками выемки засыпается щебнем.
\item Дренажные колодцы закрываются крышками.

\end{enumerate}

\end{document} % конец документа

