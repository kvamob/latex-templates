\textbf{Трещинные воды}. Уральская сложная гидрогеологическая складчатая область располагается в пределах орографически выраженного одноименного горно - складчатого сооружения. Основными коллекторами подземных вод являются трещиноватые породы коренного субстрата. Мощность зоны региональной трещиноватости составляет в среднем 30  --  100~м. Минимальные значения (20  --  30~м) присущи корам выветривания интрузивных пород; максимальные (60  --  100~м)  ---  карбонатным породам; средние (40  --  60~м)  ---  эффузивно-осадочным и метаморфическим комплексам. Помимо трещин выветривания широко развиты локальные линейные трещинные зоны высокой проницаемости и водоотдачи, связанные с проявлениями дизъюнктивной тектоники и контактами разнородных пород. Подземные воды региональной трещиноватости обычно гидравлически взаимосвязаны, имеют безнапорный характер и образуют небольшие бассейны с интенсивным
водообменном. В вертикальном разрезе фильтрационные свойства пород зоны выветривания неоднородны. По характеру их изменения зона разделяется на три части. В верхней (10  --  20~м), где широко представлены глины или суглинки элювиальной коры выветривания, водопроницаемые свойства очень низки; особенно широко коры распространены на площади Зауральского пенеплена. Средняя часть эрозионной зоны отличается наибольшей активностью, сильной степенью трещиноватости и высокой пористостью (от 1 до 7 \%). В нижней части размеры трещин весьма незначительны, и водоотдача пород практически отсутствует.

\textit{Водоносная зона трещиноватости позднепалеозойских интрузивных кислых и средних пород} ($\gamma PZ_3$) связана с массивами гранитов, гранодиоритов, плагиогранитов, диоритов. Региональная зона выветривания на этих массивах не превышает 15  --  20~м, зеркало подземных вод в сглаженной форме повторяет современный рельеф. Водоносность зоны крайне неравномерна: в центральных частях массивы практически безводны; по периферии (в приконтактовых частях с другими породами) водоносность возрастает до 0,2  --  0,3~л/с. В частности в тектонических и приконтактовых зонах позднепалеозойского Верхисетского гранитного массива отдельные скважины имели дебит до 7,0~л/с при удельном дебите 2,4~л/с, в центре массива оставаясь безводными. Минерализация вод  ---  в пределах 0,08 -- 0,5 г/дм\textsuperscript{3}; по составу преобладают гидрокарбонатные  кальциево-магниевые.

Гранодиориты и плагиограниты становятся более водоносными за счет секущих позднепалеозойских даек и жил «обновленных», к тому же тектоническими движениями создавшими условия для локализации трещинно-жильных вод в зоне выветривания. В окраинных частях Сысертского массива гранодиоритов и плагиогранитов дебиты скважин изменяются от 1,5 до 10,0~л/с при удельных дебитах 0,1  --  8,0~л/с. В центральной части (где нет жильных тел) в маломощной зоне выветривания дебиты скважин не превышают 0,3  --  0,5~л/с при удельных дебитах 0,001 -- 0,01~л/с. Ближе к периферии фиксируютя мелкие источники с расходом 0,1  --  0,2~л/с. По минерализации и химическому составу воды аналогичны развитым в гранитных массивах. 

В целом питание подземных вод Большеуральского бассейна сезонное, за счет инфильтрации атмосферных осадков в теплый период года. Зеркало его вод в сглаженной форме повторяет основные элементы рельефа. На склонах и уплощенных водоразделах уровни воды залегают на глубинах 5  --  20~м, на хребтах и локальных возвышенностях  ---  до 30  --  50~м. Сравнительно глубокая расчлененность дневной поверхности, особенно в районах приподнятых горных массивов, обеспечивает хорошие условия дренирования водоносных зон речной сетью. Разгрузка вод идет преимущественно вдоль долин рек, а также может быть приурочена к локальным трещинным зонам. Дебиты родников, в зависимости от величины водосборной площади, варьируют от долей до многих десятков литров в секунду.

Эксплуатационные ресурсы Большеуральского бассейна связываются преимущественно с крупными карбонатными массивами среднего и верхнего палеозоя и тектонически активными зонами разломов, на которых возможна организация водозаборов с дебитом 100  --  1000~л/с. На остальных водоносных зонах трещиноватости возможен каптаж подземных вод по отдельным кустам скважин с дебитами от 10 до 30~л/с. Ресурсы порово-пластовых вод Иртыш-Обского бассейна связаны преимущественно с опоковым горизонтом палеоцена, мощность обводненной толщи которого 3  --  4~м на западе и до 50~м на востоке. На придолинных участках некоторые водозаборы из него имеют производительность до 50 -- 100~л/с.

Несмотря на значительное количество водоносных горизонтов на изученной площади, наиболее промышленные и обжитые районы Урала испытывают недостаток в водообеспеченности из-за неравномерного распределения ресурсов, а также в связи с невозможностью создания крупных водозаборов с высокой производительностью на локальных участках водоносных зон.
