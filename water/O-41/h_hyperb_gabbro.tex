\textbf{Трещинные воды}. Уральская сложная гидрогеологическая складчатая область располагается в пределах орографически выраженного одноименного горно - складчатого сооружения. Основными коллекторами подземных вод являются трещиноватые породы коренного субстрата. Мощность зоны региональной трещиноватости составляет в среднем 30 – 100 м. Минимальные значения (20 – 30 м) присущи корам выветривания интрузивных пород; максимальные (60 – 100 м) – карбонатным породам; средние (40 – 60 м) – эффузивно-осадочным и метаморфическим комплексам. Помимо трещин выветривания широко развиты локальные линейные трещинные зоны высокой проницаемости и водоотдачи, связанные с проявлениями дизъюнктивной тектоники и контактами разнородных пород. Подземные воды региональной трещиноватости обычно гидравлически взаимосвязаны, имеют безнапорный характер и образуют небольшие бассейны с интенсивным водообменном. В вертикальном разрезе фильтрационные свойства пород зоны выветривания неоднородны. По характеру их изменения зона разделяется на три части. В верхней (10 – 20 м), где широко представлены глины или суглинки элювиальной коры выветривания, водопроницаемые свойства очень низки; особенно широко коры распространены на площади Зауральского пенеплена. Средняя часть эрозионной зоны отличается наибольшей активностью, сильной степенью трещиноватости и высокой пористостью (от 1 до 7 \%). В нижней части размеры трещин весьма незначительны, и водоотдача пород практически отсутствует.

\textit{Водоносная зона трещиноватости нижнепалеозойских ультраосновных пород} ($\Sigma PZ_1$)
связана с перидотитами, дунитами и полнопроявленными серпентинитами, образующими в рельефе значительные возвышенности субмеридионального простирания с ограниченными бассейнами питания трещинных грунтовых вод. Породы весьма устойчивы к процессам выветривания; мощности трещиноватой зоны не превышают 10 – 15 м. В центральных частях массивы практически безводны, а в окраинных расход родников варьирует от 0,01 до 0,2 – 0,3 л/с. Дебиты скважин, вскрывших выветрелые трещиноватые серпентиниты, не превышают 1,5 – 2,5 л/с.

Наибольшая обводненность приурочена к периферийным разломам, обновленным неотектоникой. В частности, по отдельным данным, восточная часть Колинского серпентинитового массива (в районе г. Серов) «срезана» в плане и опущена на 200 м региональным ступенчатым сбросом, прослеживаемым на 150 км. Вдоль этого нарушения ультрамафиты трещиноваты и аккумулируют подземный сток грунтовых вод зоны выветривания. Дебиты
скважин в этой зоне изменяются от 5 до 30 л/с при удельных дебитах 0,5 – 6,5 л/с. По периферии других массивов водоносность нередко связана с жилами и дайками кислого и основного составов, имеющими более молодой возраст. Грунтово-трещинные и трещинно-жильные воды имеют минерализацию 0,1–0,5 г/дм\textsuperscript{3}, и лишь на отдельных массивах встречаются ультрапресные воды. По химическому составу они гидрокарбонатные магниевые или гидрокарбонатные магниево-кальциевые. Высокие показатели магния обусловлены большим содержанием его окиси в коренных породах.

Водоносные горизонты зон высокой проницаемости и водоотдачи приурочены к омоложенным в новейшее время дизъюнктивам. Детально одна из таких тектонических зон была изучена А. В. Скалиным в Екатеринбурге
при инженерно-геологических изысканиях под высотные здания в центре города. Она приурочена к контакту габбрового массива с вулканогенной толщей и имеет ширину до 60 м. Коэффициент фильтрации здесь составляет 10 – 20 м/сут (трещиноватое габбро – до 1 м/сут). При опытных откачках дебит скважин в трещиноватых габбро не превышает 1,4 дм\textsuperscript{3}/с, в тектонической зоне – 3,5 дм\textsuperscript{3}/с. Суммарный дебит последней по кустовой откачке трех скважин составил 950 м\textsuperscript{3}/сут.

В целом питание подземных вод Большеуральского бассейна сезонное, за счет инфильтрации атмосферных осадков в теплый период года. Зеркало его вод в сглаженной форме повторяет основные элементы рельефа. На склонах и уплощенных водоразделах уровни воды залегают на глубинах 5 – 20 м, на хребтах и локальных возвышенностях – до 30 – 50 м. Сравнительно глубокая расчлененность дневной поверхности, особенно в районах приподнятых горных массивов, обеспечивает хорошие условия дренирования водоносных зон речной сетью. Разгрузка вод идет преимущественно вдоль долин рек, а также может быть приурочена к локальным трещинным зонам. Дебиты родников, в зависимости от величины водосборной площади, варьируют от долей до многих десятков литров в секунду.

Эксплуатационные ресурсы Большеуральского бассейна связываются преимущественно с крупными карбонатными массивами среднего и верхнего палеозоя и тектонически активными зонами разломов, на которых возможна организация водозаборов с дебитом 100 – 1000 л/с. На остальных водоносных зонах трещиноватости возможен каптаж подземных вод по отдельным кустам скважин с дебитами от 10 до 30 л/с. Ресурсы порово-пластовых вод Иртыш-Обского бассейна связаны преимущественно с опоковым горизонтом палеоцена, мощность обводненной толщи которого 3 – 4 м на западе и до 50 м на востоке. На придолинных участках некоторые водозаборы из него имеют производительность до 50–100 л/с.

Несмотря на значительное количество водоносных горизонтов на изученной площади, наиболее промышленные и обжитые районы Урала испытывают недостаток в водообеспеченности из-за неравномерного распределения ресурсов, а также в связи с невозможностью создания крупных водозаборов с высокой производительностью на локальных участках водоносных зон.
