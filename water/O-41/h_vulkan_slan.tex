\textbf{Трещинные воды}. Уральская сложная гидрогеологическая складчатая область располагается в~пределах орографически выраженного одноименного горно -- складчатого сооружения. Основными коллекторами подземных вод являются трещиноватые породы коренного субстрата. Мощность зоны региональной трещиноватости составляет в~среднем 30 –- 100~м. Минимальные значения (20 -- 30~м) присущи корам выветривания интрузивных пород; максимальные (60  -- 100~м)  --- карбонатным породам; средние (40 -- 60~м) --- эффузивно-осадочным и~метаморфическим комплексам. Помимо трещин выветривания широко развиты локальные линейные трещинные зоны высокой проницаемости и~водоотдачи, связанные с проявлениями дизъюнктивной тектоники и~контактами разнородных пород. Подземные воды региональной трещиноватости обычно гидравлически взаимосвязаны, имеют безнапорный характер и~образуют небольшие бассейны с интенсивным водообменном. В вертикальном разрезе фильтрационные свойства пород зоны выветривания неоднородны. По характеру их изменения зона разделяется на~три части. В верхней (10 -- 20~м), где широко представлены глины или суглинки элювиальной коры выветривания, водопроницаемые свойства очень низки; особенно широко коры распространены на~площади Зауральского пенеплена. Средняя часть эрозионной зоны отличается наибольшей активностью, сильной степенью трещиноватости и~высокой пористостью (от 1 до~7~\%). В нижней части размеры трещин весьма незначительны, и~водоотдача пород практически отсутствует.

\textit{Водоносная зона трещиноватости палеозойских вулканогенно -- осадочных и~метаморфических образований} ($an, gsPR_1–PZ$) представлена эффузивами различного состава и~их туфами, туфоконгломератами и~туфопесчаниками различного состава при участии метаморфических зеленых аповулканогенных и~глинисто-кремнистых сланцев, слюдяных кристаллосланцев, прослоями метапесчаников и~метаморфизованных известняков. Вулканогенные породы и~метаморфиты обладают близкими коллекторскими свойствами; подземные воды в~них приурочены к~грубообломочным и~карбонатным прослоям, а также к~зонам трещиноватости тектонического происхождения. 

Фильтрационное поле метаморфических пород расчленяется на~серию гидравлически связанных и~вытянутых по~простиранию блоков шириной от десятков до~первых километров, структура которых имеет мозаичный характер, обусловленный сочетанием трещин регионального выветривания, разломов и~линейных слабопроницаемых экранов. Водоносная трещиноватая зона прослеживается до~глубин от 30 до~100~м, преобладающие значения --- до~40 -- 60~м.

Наиболее глубоко она проникает в~горной (открытой) части Уральского складчатого сооружения; близко от поверхности фиксируется в~тектонически ослабленных зонах и~в~долинах рек. На площади Зауральского пенеплена водоносный горизонт перекрыт глинистой корой выветривания и~имеет напорный характер до~10 -- 15~м. Водопритоки в~скважины составляют от 0,1  -- 0,5 до~2  -- 3~л/с; в~локальных трещинах  --- в~5–10 раз выше. Дебиты родников в~доли нах рек варьируют от 0,5 до~15  -- 20~л/с (средний расход 0,8~л/с).
Высокой водообильностью обладают жильные тела, секущие вулканогенно-осадочные породы: дебиты скважин в~них составляют 2–3~л/с. На Березовском золоторудном поле, которое представляет собой огромное скопление жил (площадью около 75~км\textsuperscript{2}), шахтный водоотлив в~годы наиболее интенсивных горных работ составлял 500  -- 600~м\textsuperscript{3}/ч. Питание подземных вод происходит за счет атмосферных осадков; по~химическому составу воды гидрокарбонатные кальциево-магниевые, с минерализацией 0,2  -- 0,6~г/дм\textsuperscript{3}.

В целом питание подземных вод Большеуральского бассейна сезонное, за счет инфильтрации атмосферных осадков в~теплый период года. Зеркало его вод в~сглаженной форме повторяет основные элементы рельефа. На склонах и~уплощенных водоразделах уровни воды залегают на~глубинах 5  -- 20~м, на~хребтах и~локальных возвышенностях  --- до~30  -- 50~м. Сравнительно глубокая расчлененность дневной поверхности, особенно в~районах приподнятых горных массивов, обеспечивает хорошие условия дренирования водоносных зон речной сетью. Разгрузка вод идет преимущественно вдоль долин рек, а также может быть приурочена к~локальным трещинным зонам. Дебиты родников, в~зависимости от величины водосборной площади, варьируют от долей до~многих десятков литров в~секунду.

Эксплуатационные ресурсы Большеуральского бассейна связываются преимущественно с крупными карбонатными массивами среднего и~верхнего палеозоя и~тектонически активными зонами разломов, на~которых возможна организация водозаборов с дебитом 100 -- 1000~л/с. На остальных водоносных зонах трещиноватости возможен каптаж подземных вод по~отдельным кустам скважин с дебитами от 10 до~30~л/с. Ресурсы порово--пластовых вод Иртыш--Обского бассейна связаны преимущественно с опоковым горизонтом палеоцена, мощность обводненной толщи которого 3 -- 4~м на~западе и~до~50~м на~востоке. На придолинных участках некоторые водозаборы из~него имеют производительность до~50 -- 100~л/с.

Несмотря на~значительное количество водоносных горизонтов на~изученной площади, наиболее промышленные и~обжитые районы Урала испытывают недостаток в~водообеспеченности из-за неравномерного распределения ресурсов, а также в~связи с невозможностью создания крупных водозаборов с высокой производительностью на~локальных участках водоносных зон.
