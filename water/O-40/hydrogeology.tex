% Этот шаблон документа разработан в 2017 году
% Владимиром Коротковым (kvamob@mail.ru) 
%  

\documentclass[a4paper,12pt,twoside]{article} % добавить leqno в [] для нумерации слева

%\usepackage[14pt]{extsizes} % Задать 14-ый размер шрифта

%% Глобальные Параметры страницы
\usepackage[left=3cm,right=2cm,top=1cm,bottom=2cm,bindingoffset=0cm]{geometry}

%\includeonly{O-41/h_granite.tex}

%\usepackage{fp} 						% Вычисления с плавающей точкой
%\usepackage{siunitx}					% При использовании пакета fp все числа должны иметь decimal delimiter точку
%\sisetup{output-decimal-marker={,}}	% Числа выводятся с запятой в качестве разделителя разрядов: \num{3.2} выводит 3,2 

%%% Работа с русским языком
\usepackage{cmap}					% поиск в PDF
\usepackage{mathtext} 				% русские буквы в формулах
\usepackage[T2A]{fontenc}			% кодировка
\usepackage[utf8]{inputenc}			% кодировка исходного текста
\usepackage[english,russian]{babel}	% локализация и переносы
\usepackage{indentfirst}			% отступы даже в первом абзаце
\frenchspacing


%%% 
%\usepackage{rotating}				% Поворот текста

%%% Дополнительная работа с математикой
\usepackage{amsmath,amsfonts,amssymb,amsthm,mathtools} % AMS
\usepackage{icomma} % "Умная" запятая: $0,2$ --- число, $0, 2$ --- перечисление
\usepackage{gensymb}	% Градусы
%% Номера формул
%\mathtoolsset{showonlyrefs=true} % Показывать номера только у тех формул, на которые есть \eqref{} в  тексте.

%% Шрифты
\usepackage{euscript}	 % Шрифт Евклид
\usepackage{mathrsfs} % Красивый матшрифт


%% Перенос знаков в формулах (по Львовскому)
% \newcommand*{\hm}[1]{#1\nobreak\discretionary{}
%	{\hbox{$\mathsurround=0pt #1$}}{}}

%%% Работа с картинками
\usepackage{graphicx}  % Для вставки рисунков
%\usepackage[export]{adjustbox}
\graphicspath{{images/}}  % папки с картинками
\setlength\fboxsep{3pt} % Отступ рамки \fbox{} от рисунка
\setlength\fboxrule{0.2pt} % Толщина линий рамки \fbox{}
\usepackage{wrapfig} % Обтекание рисунков и таблиц текстом


%%% Работа с таблицами
\usepackage{array,tabularx,tabulary,booktabs} % Дополнительная работа с таблицами
\usepackage{longtable}  % Длинные таблицы
\usepackage{multirow} % Слияние строк в таблице

%%% Подписи к рисункам и таблицам в русской типографской традиции
\usepackage{caption} 
\DeclareCaptionFormat{GOSTtable}{#2#1\\#3}
\DeclareCaptionLabelSeparator{fill}{\hfill}
\DeclareCaptionLabelSeparator{dot}{. }
\DeclareCaptionLabelFormat{fullparents}{\bothIfFirst{#1}{~}#2}
\captionsetup[table]{
	format=GOSTtable,
	font={footnotesize},
	labelformat=fullparents,
	labelsep=fill,
	labelfont=rm,
%	labelfont=it,
	textfont=bf,
	justification=centering,
	singlelinecheck=false
}
\captionsetup{font=small}
\captionsetup[figure]{
	labelsep=dot, 
%	textfont=it
}
% А можно и так
%\captionsetup{labelsep=period}
\captionsetup{listfigurename=Список рисунков}		% По умолчанию список иллюстраций


%%% Модификация команд, задающих разделы
% Не подавлять отступы у первого абзаца

\makeatletter   % Команда \makeatletter делает символ @ буквой, команда \makeatother возвращает всё на свои места.
% Разрешим отступ у первого абзаца
\renewcommand\section{\@startsection {section}{1}{\parindent}%
	{3.5ex \@plus 1ex \@minus .2ex}{2.3ex \@plus.2ex}%
	{\normalfont\hyphenpenalty=10000\Large\bfseries}}

\renewcommand\subsection{\@startsection {subsection}{1}{\parindent}%
	{3.5ex \@plus 1ex \@minus .2ex}{2.3ex \@plus.2ex}%
	{\normalfont\hyphenpenalty=10000\large\bfseries}}
\makeatother

% После номеров разделов \section ставить точки
\usepackage{secdot}			
% И после \subsection тоже ставить точки
\sectiondot{subsection}		

\usepackage{lastpage} % Узнать, сколько всего страниц в документе.

\usepackage{soulutf8} % Модификаторы начертания

\usepackage{hyperref}
%\usepackage[usenames,dvipsnames,svgnames,table,rgb]{xcolor}
\hypersetup{				% Гиперссылки
	unicode=true,           % русские буквы в раздела PDF
	%	pdftitle={Заголовок},   % Заголовок
	%	pdfauthor={Автор},      % Автор
	%	pdfsubject={Тема},      % Тема
	%	pdfcreator={Создатель}, % Создатель
	%	pdfproducer={Производитель}, % Производитель
	%	pdfkeywords={keyword1} {key2} {key3}, % Ключевые слова
	colorlinks=true,       	% false: ссылки в рамках; true: цветные ссылки
	linkcolor=black,          % внутренние ссылки
	%	linkcolor=red,          % внутренние ссылки
	%	citecolor=green,        % на библиографию
	%	filecolor=magenta,      % на файлы
	urlcolor=cyan           % на URL
}

\usepackage{multicol} % Несколько колонок

\usepackage{fancyhdr} % Колонтитулы
\pagestyle{fancy}
%\renewcommand{\headrulewidth}{0mm}  % Толщина линейки, отчеркивающей верхний колонтитул
%\lfoot{Нижний левый}
%\rfoot{Нижний правый}
%\rhead{Верхний правый}
%\chead{Верхний в центре}
%\lhead{Верхний левый}
% \cfoot{Нижний в центре} % По умолчанию здесь номер страницы
%
% Нумерация страниц внизу справа
\fancyhf{}
\fancyfoot[R]{\thepage}		
\renewcommand{\headrulewidth}{0pt}
\renewcommand{\footrulewidth}{0pt}

%%%%%%%%%%%%%%%%%%%%%%%%%%%%%%%%%%%%%%%%%%%%%%%%%%%%%%%%%%%%%%%%%%%%%%%%%%%%%%%%%%%%%%%%%%%%%%%%%%%%%%%%%

\begin{document} % конец преамбулы, начало документа

\section*{ЛИСТ O-40}
\section*{ГИДРОГЕОЛОГИЯ}
стр.372 в пояснительной записке к карте 0-40

Ниже приводится краткая характеристика основных гидрогеологических
подразделений зоны активного водообмена, имеющих распространение по
площади с выходом на поверхность и имеющих практическое значение.

Первыми от поверхности и выдержанными по площади гидрогеологическими объектами в долинах крупных водотоков и на прилегающих водораздельных участках являются водоносные комплексы озерно -- аллювиальных
отложений и ледниковых и водно -- ледниковых отложений.

\textit{Неоплейстоцен -- голоценовый водоносный горизонт} озерно -- аллювиальных отложений ($laQI - H$) распространен в долинах современных рек, с широким развитием в долинах рек Кама, Чусовая, Сылва, Уфа. Его мощность находится обычно в пределах 5 -- 15 м (до 32 м в долине р. Кама). Химический
состав подземных вод преимущественно гидрокарбонатный кальциевый, магниево -- кальциевый, реже натриевый с минерализацией 0,1 -- 0,5 (в среднем
0,15~г/дм\textsuperscript{3}). Из -- за низкого гипсометрического положения, помимо болотного загрязнения ($Na, Cl, SO_4, NO_3$), велика вероятность попадания в него
разного рода сточных вод. Воды горизонта используются для водоснабжения водозаборами. Из -- за гидравлической связи вод горизонта с поверхностными
водами, которые часто загрязнены, сооружать водозаборы в пределах данного горизонта не рекомендуется.

\textit{Нижне -- среднеплейстоценовый водоносный горизонт ледниковых и водно -- ледниковых отложений} ($q,flQI - II$) выделяется ограниченно в северной
части территории листа О -- 40 в области развития ледниковых и водно -- ледниковых отложений. Мощность горизонта --- до 15 м. 
Водовмещающими породами являются озерные глинистые сапропели с растительным детритом и торф, залегающие на водоупорном плотике из ледниковых тиллов. Мощность --- 2 -- 3 м. Питание осуществляется за счет атмосферных осадков. Воды пресные, преимущественно гидрокарбонатные кальциевые, хотя отмечаются воды хлоридно -- сульфатные, что связано с болотным загрязнением. Обводненность торфов зависит от метеорологических условий. Разгрузка вод осуществляется путем перетекания в нижележащие водоносные комплексы.

\section*{Уральский бассейн зон трещиноватости (III)}

В его состав входят три бассейна: Уральский, Западно -- Уральский и Ляпинский \textbf{(граф. прил. 6)}. Для этой области характерно двухъярусное строение. Верхний гидрогеодинамический этаж охватывает зону экзогенной трещиноватости, где развиты трещинный, трещинно -- карстовый и спорадически
блоково -- порово -- пластовый типы подземных вод, относящихся по условиям залегания, циркуляции и характеру водообмена с поверхностными водами к
водам зоны активного водообмена. Водообильность пород обеспечивается тем, что они широко развиты на дневной поверхности, имея тем самым обширную область питания за счет инфильтрации атмосферных осадков; кроме того, приток вод происходит по трещинам из других водоносных горизонтов.
Интенсивному водообмену различных горизонтов способствует дислоцированность пород. Мощность верхнего этажа наибольших значений достигает в
водораздельной части. Нижний гидродинамический этаж охватывает часть разреза с жильно -- блоковым типом водовмещающих тел ниже зоны экзоген
ной трещиноватости, содержащую напорные воды, связанные преимущественно с тектоническим разломами.

\textit{Девонско -- нижнепермская водоносная зона трещиноватости Язьвинско -- Чусовской складчато -- надвиговой зоны} ($D - P_1$). Сложена терригенно -- 
карбонатными (песчаники, алевролиты, аргиллиты, известняки, доломиты, глинистые известняки) отложениями мощностью 1300 -- 3400 м.
Узколинейный характер многочисленных аллохтонов, осложняющих их складок, частое чередование выходов карбонатных и терригенных толщ,
а также наличие разрывов взбросово -- надвигового типа обусловливают линейность и локальность развития водоносных горизонтов. Наиболее обводнены трещиноватые, пористые, интенсивно -- карстующиеся карбонаты. 

С карстовыми образованиями связаны источники, характеризующиеся наибольшими расходами. Выходы многочисленных родников располагаются
в основании склонов долин водотоков, где дебит их изменяется в широких пределах (от 0,05 до 300 л/с).
Глубина залегания подземных вод зоны колеблется от 1 -- 2 м в долинах до 50 м на водоразделах. Там, где трещиноватость пронизывает всю толщу пород, уровень вод свободный, при наличии нетрещиноватых водоупорных прослоев появляется напор, достигающий в долинах 15 -- 20 м. Удельные дебиты скважин составляют 0,13 -- 5,3 л/с.
По составу воды преимущественно гидрокарбонатные слабощелочные (рН = 7,70 -- 8,30), магниево--кальциевые и натриево--кальциевые, иногда встречаются сульфатно -- гидрокарбонатные; минерализация колеблется в пределах 0,1 -- 0,7 г/дм\textsuperscript{3}.

В западной части зоны под краевыми частями аллохтонов возможно наличие соленых и рассольных вод, связанных с соленосными отложениями нижней перми.

\textit{Рифейско -- вендские водоносные зоны трещиноватости Кваркушско -- Каменногорского и Уфалейского антиклинориев} ($RF - V$) имеют достаточно
широкое распространение в восточной части территории листа О -- 40. Включают в себя подземные воды отложений клыктанской свиты, басегской, серебрянской и сылвицкой серий Кваркушско -- Каменногорского антиклинория, а также метаморфизованных образований Уфалейского антиклинория.

\textit{Водоносная зона трещиноватости клыктанских терригенно -- карбонатных образований}  включает мраморизованные известняки и доломиты с прослоями сланцев в ядрах антиклинальных структур Кваркушско--Каменногорского антиклинория. Воды трещинно -- карстовые. Территории распространения карбонатных пород отвечают понижениям в рельефе. Водоносность пестрая из -- за неравномерной трещиноватости и закарстованности.
Площади распространения совпадают с площадями питания. Помимо атмосферных осадков и сезонных вод определенное значение имеют смежные водоносные комплексы. Дебиты родников  ---  1,5 -- 2 л/с, иногда до 20 л/с, а в пластовых выходах  ---  до 40 л/с. Дебиты скважин в районе пос. Бисер  ---  2 -- 4 л/с.
Состав вод  ---  гидрокарбонатно -- магниево -- кальциевый с минерализацией 0,1 -- 0,3 г/дм\textsuperscript{3}, жесткость  --  до 3 -- 4 мг -- экв.

\textit{Водоносная зона трещиноватости рифейских образований}, представленных породами ослянской, щегровитской, федотовской и усьвинской свитами . Образования распространены в осевой части Кваркушско--Каменногорского антиклинория.

Разрез сложен преимущественно терригенными образованиями с подчиненным количеством пачек вулканогенных пород и известняков. Водовмещающими породами являются сланцы различного состава, реже мраморизованные известняки, в меньшей степени  --  кварцитопесчаники. Мощность водоносного горизонта определяется глубиной развития эффективной трещиноватости, колеблющейся в пределах 50 -- 60 м. Глубже указанных пределов
породы безводны, за исключением трещиноватых зон тектонических нарушений.

Подземные воды горизонта в теплое время имеют единый, часто свободный уровень. Водообильность горизонта в целом невелика. Сланцевые толщи
при прочих равных условиях повсеместно обводнены слабее по сравнению с другими литологическими разностями коллекторов.

Дебиты их родников достигают 0,5 л/с. Подземные воды в песчаниках, основных эффузивах и туфогенных породах обеспечивают дебиты родников до
2 л/с, но основная масса также имеет дебиты до 0,5 л/с. Водоносный горизонт в кварцитах и кварцитопесчаниках ослянской свиты имеет в целом очень небольшую мощность и слабообводнен. Водообильность его оценивается до 0,5 л/с. Повышенно обводненными являются во многих случаях приконтактовые зоны габбродолеритовых даек как в эффузивах, так и в сланцах. Родники с дебитом до 10 л/с отмечаются в трещинных зонах, ослабленных новейшими подвижками. Минерализация подземных вод меняется от 0,02 до 0,21 г/дм\textsuperscript{3}. Общая жесткость подземных вод изменяется в пределах 0,18 -- 2,83 мг -- экв, рН составляет 5,4 -- 7,8.

Запасы вод в целом незначительны, достаточны для удовлетворения хозяйственно -- питьевых нужд мелких населенных пунктов. Возможна организация водозаборов с производительностью до 2 л/с, а при заложении скважин в водоносных зонах  ---  до 5 -- 10 л/с.

\textit{Водоносная зона трещиноватости верхнерифейско -- вендских} терригенных образований, представленных верхнерифейскими (ослянская, федотовская, усьвинская свиты) и вендскими (сылвицкая и серебрянская серии) существенно терригенными комплексами пород, измененными до
зеленосланцевой ступени метаморфизма. Комплекс имеет достаточно широкое распространение в восточной части листа О -- 40 и сложен преимущественно терригенными породами со стратифицированными комплексами вулканитов. Воды комплекса, как правило, трещинные и трещинно -- жильные.
Обводненность с глубиной снижается. Воды безнапорные. Большинство родников приурочено к зонам контактов и тектонических нарушений. Питание сезонное, за счет инфильтрующихся атмосферных осадков. Разгрузка происходит в виде мелких родников, часто в долинах рек, и в виде сплошных
пластовых просачиваний, питающих мелкие ручьи и речки. Дебиты родников ---  0,05 -- 0,8 л/с, реже до 5 л/с. Дебиты скважин  ---  0,2 -- 7,5 л/с при понижениях 8 -- 20 м. Преобладают гидрокарбонатные и сульфатно -- гидрокарбонатные магниево -- кальциевые воды с минерализацией 0,02 -- 0,17 г/дм\textsuperscript{3}, реже до 0,2 г/дм\textsubscript{3}. Жесткость  ---  0,1 -- 3 мг--экв; рН  ---  5,7 -- 8,5. Горизонт может служить источником водоснабжения небольших населенных пунктов и обеспечить производительность скважин до 2 -- 3 л/с при отличном качестве отбираемых
вод. В водоносных зонах возможна организация водозаборов с производительностью 5 -- 10 л/с.

\textit{Водоносная зона трещиноватости верхнерифейско -- нижневендских вулканогенно -- терригенных метаморфизованных образований} включает
породы щегровитского, вильвенского и дворецкого вулканических комплексов. Воды трещинные и трещинно -- жильные, как правило, безнапорные. Глубина залегания подземных вод определяется мощностью зоны эффективной трещиноватости и соответствует 50 -- 60 м. Водообильность в целом незначительна, дебиты родников редко достигают 0,5 л/с. Повышенная водообильность наблюдается в контактовых зонах даек габбродолеритов, вблизи стратиграфических границ и разрывов. Удельные дебиты скважин  ---  0,02 -- 2,5 л/с, коэффициент фильтрации  ---  1,7 -- 5,9 м/сут. Дебит родников  ---  0,1 -- 0,2 л/с. Преобладают гидрокарбонатные и сульфатно -- гидрокарбонатные натриевые, кальциевые и магниевые воды с минерализацией 0,02 -- 0,21 г/дм\textsuperscript{3}, рН = 5,40 -- 7,80, общая жесткость  ---  0,18 -- 2,83 мг--экв. Изученность комплекса недостаточна.

\textit{Водоносная зона трещиноватости вендских терригенных образований} включает отложения серебрянской и сылвицкой серий венда, широко распространенных в пределах Кваркушско -- Каменногорского антиклинория.
Вендские образования представлены аргиллитами, алевролитами, песчаниками, редкогалечными конгломератами. Редкие карбонатные прослои имеют
ограниченное распространение. Воды трещинно -- грунтовые и жильные, безнапорные. Чередование слоев различной проницаемости при сложной складчатости, а также экранирование элювиально -- делювиальными образованиями приводят к появлению местного напора в придолинных участках. Водообильность слабая и неравномерная. Дебиты родников  ---  от 0,2 -- 0,3 до 2 л/с, в зонах контактов и тектонических нарушений  ---  до 10 л/с. Грунтовые потоки, благодаря глубокому расчленению рельефа обладают большими уклонами.
Основные источники питания  ---  атмосферные и талые воды, реже конденсационные. Питание затруднено из -- за повсеместного чехла перекрывающих
рыхлых отложений. Обводненность с глубиной снижается из -- за уменьшения трещиноватости, наибольшая обводненность  --  до глубины 40 -- 60 м. Дебиты
скважин  ---  0,01 -- 2,5 л/с. Преобладают сульфатно -- гидрокарбонатные кальциевые, натриевые и магниевые воды с минерализацией до 0,1 г/дм\textsuperscript{3}, редко до 0,2 -- 0,4 г/дм\textsuperscript{3}; жесткость  --  0,2 -- 2 мг -- экв, рН = 5,6 -- 7,6.

\textit{Водоносная зона трещиноватости вулканогенных и осадочных образований амфиболитовой фации метаморфизма Уфалейского антиклинория.} Комплекс получил широкое распространение в границах Уфалейской структурно -- фациальной зоны, представленной различными зелеными, графит -- слюдяными и слюдистыми сланцами, кварцитами, амфиболитами указарской и куртинской свит; метабазальтами, сланцами и амфиболитами, различными гнейсами и кварцитами уфалейского комплекса.

Мощность зоны региональной трещиноватости, равная мощности выделенных выше водоносных зон, составляет 30 -- 100 м. Минимальные ее значения (30 -- 40 м) присущи корам выветривания интрузивных пород, максимальные (60 -- 100 м)  ---  карбонатным породам. В породах эффузивно -- осадочного и метаморфического комплексов она оценивается в 40 -- 60 м.

Помимо трещин выветривания широким развитием пользуются локальные линейные трещинные зоны аномально высокой проницаемости и водоотдачи,
связанные с проявлениями дизъюнктивной тектоники, внедрением интрузий, контактами карстующихся пород с некарстующимися. Открытая трещиноватость в этих зонах прослеживается вглубь на многие сотни метров.

По водоотдаче перечисленные выше водоносные зоны отличаются друг от друга. Водопритоки в скважины, вскрывшие кору выветривания интрузивных
пород, имеют дебиты менее 0,5 л/с, при максимальных понижениях уровня относительно статического. Водопритоки в скважины в карбонатных породах  --- 1--10 л/с, в метаморфических, вулканогенных и терригенных породах ---  от 0,1--0,5 до 2--3 л/с. В локальных трещинных зонах водопритоки в скважины в 5 -- 10 раз и более превышают фоновые значения. Например, скважина, вскрывшая зону разлома в известняках нижнего карбона в долине р. Уфа
(устье р. Табуска), имела дебит 84 л/с при понижении уровня воды 2,3 м; скважины, вскрывшие разломы в известняках нижнего девона в долине
р. Серга, имели дебиты до 150 л/с. 

Уровни подземных вод в сглаженной форме повторяют основные элементы рельефа. На склонах и хорошо выраженных водораздельных пространствах они залегают на глубинах 10 -- 20 м, а на отдельных участках (районы Бардымского горного сооружения)  ---  до 50 м. В широких плоских речных долинах глубина залегания уровней воды измеряется долями метра или первыми метрами.

Питание подземных вод сезонное, за счет инфильтрации атмосферных осадков в теплый период года.

\textit{Средне -- верхнеордовикская водоносная зона трещиноватости (промысловская серия Безгодовского клиппа  --  $O_{2-3}pr$)} имеет место в юго -- западной части территории листа О -- 40 -- XI. Комплекс представлен терригеннокарбонатными породами промысловской серии. Подземные воды комплекса заключены в прослоях и пачках песчаников, известняков и доломитов, разобщенных глинистыми сланцами. Водообильность комплекса в целом невелика. Дебиты родников изменяются от 0,05 -- 2 л/с, в отдельных случаях до 10 -- 25 л/с, при разгрузке в совокупности с водами элювиально -- делювиальных
отложений. Многочисленные разрывные нарушения гидрогеологически себя не проявляют. Минерализация подземных вод комплекса  ---  до 0,1 г/дм\textsuperscript{3}. Запасы подземных вод комплекса незначительны, перспективы их использования ограничены.
Силурийско--девонская водоносная зона трещиноватости Бардымско--Нязепетровского аллохтона ($PZ_1$). Режим грунтовых вод водоносной зоны
полностью отражает условия их питания и геоморфологического положения отдельных участков. При нормальной летней водности восполнение
ресурсов подземных вод продолжается до конца августа, затем идет медленный спад, продолжающийся до начала весны следующего года. Амплитуда
между минимальным и максимальным положениями уровня воды в долинах рек оценивается в 1 -- 1,5 м, на склонах водоразделов и на самих водоразделах ---  от 1,5 -- 5 до 10 -- 20 м и более (последняя  ---  в районах приподнятых горных массивов).

Избыточная увлажненность, хорошие условия дренирования водоносных зон при преобладающем силикатном составе водовмещающих коллекторов
обусловили формирование здесь мягких гидрокарбонатных вод с минерализацией от 0,05 -- 0,1 до 0,12 -- 0,35 г/дм\textsuperscript{3}. По катионному составу доминируют воды магниево -- кальциевые.

Трещинно -- жильные воды гидравлически тесно связаны с водами зоны региональной трещиноватости. Опробованием структурных скважин установлено, что, начиная с глубины 200 -- 300 м, гидрокарбонатные магниево--кальциевые воды верхней зоны сменяются на гидрокарбонатно--хлоридные
натриевые воды с минерализацией до 1 г/дм\textsuperscript{3}, при резком снижении воодоотдачи пород (скважины, вскрывшие известняки нижнего -- среднего девона в долине р. Демид и кремнистые породы Бардымского комплекса на участке южнее Нижне -- Сергинского пруда).
В долине р. Серга, на юго -- западной окраине г. Нижние Серги и южнее по простиранию Сергинского регионального разлома имеются выходы сероводородных хлоридных натриевых вод. Последние, как считают многие, поступают по названному разлому со стороны Киргишанского тектонического покрова, под которым, возможно, сохранились пермские соленосные осадки.

\textit{Нижнепалеозойская водоносная зона трещиноватости Улсовско -- Висимской синклинали ($PZ_1$)} распространена в виде узкой полосы, протягивающейся почти через весю территорию с севера на юг вдоль Присалатимского надвига.

Подземные воды комплекса можно подразделить на два горизонта. Первый из них связан с преимущественно терригенными породами
среднего и верхнего ордовика. Водоносными являются маломощные прослои и линзы известняков и доломитов, разделенные водоупорными глинистыми и
известняково -- глинистыми сланцами. Дебиты немногочисленных родников колеблются от 0,01 до 2 л/с, и только на участках пластовой разгрузки отдельных прослоев известняков суммарный дебит достигает 10 л/с. Минерализация  ---  до 0,17 г/дм\textsuperscript{3}.

Второй водоносный горизонт связан с существенно карбонатными породами среднеордовикско -- среднедевонского возраста центральной части Улсовско -- Висимской синклинали и образует крупнейший в пределах Центрально -- Уральского поднятия бассейн подземных вод. В связи с неглубокой эрозионной расчлененностью мощность зоны аэрации невелика. Уровень подземных вод обычно свободный и находится на глубине до 10 м. Водообильность горизонта крайне неравномерная. Срединный дебит родников колеблется в пределах 0,1 -- 25 л/с. Дебиты от десятков сотен л/с имеют родники,
фиксирующие локальные водоносные зоны. На отдельных участках, перекрытых водонепроницаемыми рыхлыми отложениями, проявляется местный
напор; в скважинах наблюдается самоизлив с глубин 3 -- 17 м с дебитом 1,5 л/с. Минерализация варьирует в пределах 0,05 -- 0,19 г/дм\textsuperscript{3}. Воды мягкие или умеренножесткие слабощелочные. Химический состав вод в основном гидрокарбонатно -- кальциево--магниевый, гидрокарбонатно--натриевый, гидрокарбонатно--хлоридно--натриевый с минерализацией до 0,30 г/дм\textsuperscript{3}, в зонах разломов
сульфатно -- натриевый с минерализацией вод 0,07 г/дм\textsuperscript{3}, pH = 5,65 -- 8,70. Температура вод  ---  2 -- 6,5 $\degreeС$.
Современное использование ресурсов горизонта ограничивается водозабором из родников для нужд населения мелких населенных пунктов. Воды
комплекса благоприятны для водоснабжения. Оборудованные скважины могут давать до 500 л/сут питьевой воды.

\textit{Ордовикско -- девонская водоносная зона трещиноватости Тагильского синклинория ($O - D$)} распространена в восточной половине листа.
Она слагает приосевую часть Тагильской структуры и представлена эффузивами различного состава и их туфами, вулканогенно -- осадочными и осадочными породами. Зона имеет сложное строение в связи с частой фациально -- литологической изменчивостью и сильной дислоцированностью. Глубина распространения трещинной зоны выветривания колеблется в пределах 30 -- 60 м. Подземные воды имеют свободный уровень, залегающий на глубинах
10 -- 15 м; на повышенных участках рельефа, сложенных вулканогенными породами  ---  до 50 м. К долинам рек наблюдается снижение зеркала подземных
вод, как и рельефа местности, в связи с дренированием их водотоками. Дебиты родников в долинах составляют до 0,5 л/с, при пересечении линейных
зон  ---  до 15 л/с. Водоносность зоны крайне неравномерна и зависит от литологического состава пород и их тектонической раздробленности. Наиболее
водообильны толщи переслаивания туфов с кремнистыми сланцами, а также линзы известняков, контакты которых омоложены неотектоническими подвижками. В этих зонах дебиты родников достигают 5 -- 10 л/с, за их пределами  ---  0,05 -- 0,3 л/с. Воды пресные, с минерализацией 0,1 -- 0,2 г/дм\textsuperscript{3}, гидрокарбонатные кальциевые. На участках эффузивов с богатой сульфидной минерализацией в подземных водах появляются сульфаты.

\textit{Нижневендско -- силурийская водоносная зона трещиноватости метаморфических образований белогорского комплекса и магматических пород
конжаковского и тагило -- кытлымского комплексов (V1 -- S1)} распространена ограниченно в северо -- восточной части площади. Водоносная зона подразделяется на зоны трещиноватости пород белогорского гнейсово -- амфиболитового метаморфического комплекса и магматических образований конжаковского и тагило -- кытлымского магматических комплексов.

\textit{Водоносная зона трещиноватости белогорского гнейсово -- амфиболитового метаморфического комплекса} сложена гнейсами и амфиболитами. 
Глубина зоны интенсивной экзогенной трещиноватости пород составляет на междуречьях 20 -- 30 м, в долинах рек  ---  до 50 м. Подземные воды приурочены преимущественно к узким зонам контактов сланцев и первично терригенных пород, а также к тектоническим нарушениям. Коэффициенты фильтрации кристаллических сланцев и гнейсов составляют десятые доли метра в сутки, в тектонически ослабленных зонах мраморов  ---  до 3 -- 5 м/сут.
Зеркало подземных вод фиксируется обычно на глубинах 10 -- 15 м и повторяет рельеф местности. Дебиты родников в долинах рек составляют десятые
доли метра в секунду, на придолинных частях  ---  до 1 -- 3 л/с; в тектонических зонах мраморизованных известняков  ---  до 10 л/с. Воды пресные, с минерализацией 0,03 -- 0,12 г/дм\textsuperscript{3}, по химическому составу гидрокарбонатные кальциевые. Отмечается дефицит содержания фтора  ---  до 0,02 г/дм\textsuperscript{3}. 

\textit{Водоносная зона трещиноватости магматических образований конжаковского и тагило -- кытлымского магматических комплексов}, которая приурочена к габброноритовому тагило -- кытлымскому и дунит -- клинопироксенит -- габбровому конжаковскому комплексам. Для них присущи общие закономерности: глубина экзогенной трещиноватости не превышает 25 -- 30 м, питание за счет атмосферных осадков, маловодность на междуречьях, сниже
ние зеркала подземных вод к долинам рек согласно рельефу, повышение обводненности на участках с увеличением более позднего внедрения в массивы
различных по составу даек, к контактам и тектоническим нарушениям. Дебиты родников в долинах рек редко превышают 0,1 -- 0,3 л/с; по периферии массивов на контактах с вмещающими породами и в зонах разломов достигают 6 -- 10 л/с.
Воды пресные и ультрапресные, с низкой минерализацией. По химическому составу для плагиогранитов и габбро характерен гидрокарбонатный
кальциево -- магниевый состав; для дунитов  ---  гидрокарбонатный магниевокальциевый.

\textit{Водоносные зоны трещиноватости тектонических нарушений}. Наряду с площадными гидрогеологическими объектами в регионе широко развиты и линейные объекты  --  зоны разрывных нарушений и связанные с ними тектонические зоны дробления. Подавляющее большинство выходящих на поверхность разрывов находятся в пределах Уральской складчатой системы.Здесь они чаще всего представляют собой пологие надвиги, разграничивающие многочисленные аллохтонные пластины и автохтонные массивы пород. В платформенной части листа разломы довольно редки и распространяются
на большую глубину, затрагивая фундамент.

Характерной особенностью разрывов является их аномально высокая, хотя зачастую и локальная, водообильность. Так, в Уральском бассейне фоновые
расходы источников из карбонатов палеозоя равны 5 -- 50 л/с, а вблизи зон крупных нарушений дебиты водопроявлений из тех же карбонатов возрастают до 7 м3/с. В складчатых областях по зонам разломов осуществляется как питание, так и разгрузка подземных вод (в зависимости от времени года), в
равнинных регионах преобладает разгрузка. По химическому составу воды преимущественно гидрокарбонатные кальциевые, кальциево -- магниевые с
минерализацией 0,01 -- 0,2, редко до 0,7 г/дм\textsuperscript{3}.

\end{document} % конец документа

