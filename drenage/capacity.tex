% Этот шаблон документа разработан в 2017 году
% Владимиром Коротковым (kvamob@mail.ru) 
%  

\documentclass[a4paper,12pt]{article} % добавить leqno в [] для нумерации слева

%% Глобальные Параметры страницы
\usepackage[left=3cm,right=2cm,
top=1cm,bottom=2cm,bindingoffset=0cm]{geometry}

%\usepackage{fp} 						% Вычисления с плавающей точкой
%\usepackage{siunitx}					% При использовании пакета fp все числа должны иметь decimal delimiter точку
%\sisetup{output-decimal-marker={,}}		% Числа выводятся с запятой в качестве разделителя разрядов: \num{3.2} выводит 3,2 

%%% Работа с русским языком
\usepackage{cmap}					% поиск в PDF
\usepackage{mathtext} 				% русские буквы в формулах
\usepackage[T2A]{fontenc}			% кодировка
\usepackage[utf8]{inputenc}			% кодировка исходного текста
\usepackage[english,russian]{babel}	% локализация и переносы

%%% 
%\usepackage{rotating}				% Поворот текста

%%% Дополнительная работа с математикой
\usepackage{amsmath,amsfonts,amssymb,amsthm,mathtools} % AMS
\usepackage{icomma} % "Умная" запятая: $0,2$ --- число, $0, 2$ --- перечисление

%% Номера формул
%\mathtoolsset{showonlyrefs=true} % Показывать номера только у тех формул, на которые есть \eqref{} в  тексте.

%% Шрифты
\usepackage{euscript}	 % Шрифт Евклид
\usepackage{mathrsfs} % Красивый матшрифт


%% Перенос знаков в формулах (по Львовскому)
% \newcommand*{\hm}[1]{#1\nobreak\discretionary{}
%	{\hbox{$\mathsurround=0pt #1$}}{}}

%%% Работа с картинками
\usepackage{graphicx}  % Для вставки рисунков
%\usepackage[export]{adjustbox}
\graphicspath{{images/}{images2/}}  % папки с картинками
\setlength\fboxsep{3pt} % Отступ рамки \fbox{} от рисунка
\setlength\fboxrule{0.2pt} % Толщина линий рамки \fbox{}
\usepackage{wrapfig} % Обтекание рисунков и таблиц текстом

%%% Работа с таблицами
\usepackage{array,tabularx,tabulary,booktabs} % Дополнительная работа с таблицами
\usepackage{longtable}  % Длинные таблицы
\usepackage{multirow} % Слияние строк в таблице

%\usepackage{caption} % подписи к рисункам в русской типографской традиции
%\DeclareCaptionFormat{GOSTtable}{#2#1\\#3}
%\DeclareCaptionLabelSeparator{fill}{\hfill}
%\DeclareCaptionLabelFormat{fullparents}{\bothIfFirst{#1}{~}#2}
%\captionsetup[table]{
%	format=GOSTtable,
%	font={footnotesize},
%	labelformat=fullparents,
%	labelsep=fill,
%	labelfont=rm,
%%	labelfont=it,
%	textfont=bf,
%	justification=centering,
%	singlelinecheck=false
%}

%%%%%%%%%%%%%%%%%%%%%%%%%%%%%%%%%%%%%%%%%%%%%%%%%%%%%%%%%%%%%%%%%%%%%%%%%%%%%%%%%%%%%%%%%%%%%%%%%%%%%%%%%
%%% PAYLOAD
%%%%%%%%%%%%%%%%%%%%%%%%%%%%%%%%%%%%%%%%%%%%%%%%%%%%%%%%%%%%%%%%%%%%%%%%%%%%%%%%%%%%%%%%%%%%%%%%%%%%%%%%%

%%% Заголовок
\author{ООО <<Гидросфера>>}\label{company}
\title{ПАСПОРТ РАЗВЕДОЧНО-ЭКСПЛУАТАЦИОННОЙ СКВАЖИНЫ}
\date{\today}
%%%======================================================================================================
\newcommand{\txtExecutor}{ООО <<Гидросфера>>}	% Исполнитель
\newcommand{\txtYear}{2017}						% Год
\newcommand{\txtAddress}{--Address--}			% Адрес
\newcommand{\txtCadaster}{--Cadaster--} 		% Кадастровый номер


%%%%%%%%%%%%%%%%%%%%%%%%%%%%%%%%%%%%%%%%%%%%%%%%%%%%%%%%%%%%%%%%%%%%%%%%%%%%%%%%%%%%%%%%%%%%%%%%%%%%%%%%%

\begin{document} % конец преамбулы, начало документа

\setlength{\extrarowheight}{1mm} % Дополнительный интервал между строками таблиц

%% Титульная страница

\begin{titlepage}
	\begin{center}
		\textbf{\txtExecutor}
		\vspace{5.5cm}
		
		{\LARGE Отчет по рекомендованным дренажным мероприятиям}
		\vspace{0.25cm}
		
		\bigskip
		
		на участке по адресу:
				
		\underline{\txtAddress}
		
		\bigskip
		Кадастровый номер \txtCadaster
		
		\vfill
	
		\bigskip
		
	\end{center}

	\vfill
	
	\newlength{\ML}
	\settowidth{\ML}{«\underline{\hspace{0.7cm}}» \underline{\hspace{2cm}}}
	\hfill
	\begin{minipage}{1.0\textwidth}
		Директор ООО <<Гидросфера>> к.г.м.н.
		\underline{\hspace{\ML}} А.\,А.~Кашкаров\\
	\end{minipage}%
	
	\bigskip
	
	\vfill
	\begin{center}
		Екатеринбург, \txtYear
	\end{center}			

	\end{titlepage}

%%%%%%%%%%%%%%%%%%%%%%%%%%%%%%%%%%%%%%%%%%%%%%%%%%%%%%%%%%%%%%%%%%%%%%%%%%%%%%%%

\section{Определение водозахватной способности дрен}

Проверка на достаточность водозахватной способности определяется соблюдением условия:

\begin{equation}\label{eq:debit}
	Q_0 \le f 
\end{equation}

	где 
	
	$Q_0$ -- дебит дрены в $м^3/сут$, 
	
	$f$ -- водозахватная способность
	
Расчет водозахватной способности производится по формулам С.К. Абрамова для вертикальных дрен:

\begin{equation}\label{eq:abramov}
	f = 130 \pi r_c l \sqrt[3]{K}
\end{equation}

	где 

	$f$ -- водозахватная способность дрены в $м^3/сут$ на одну дрену
	
	$r_c$ -- наружный радиус вертикальных дрен, м
	
	$l$ -- длина фильтра, м
	
	$K$ -- коэффициент фильтрации водоносного пласта в $м^3/сут$
	
	\bigskip
	
	При $r_c = 0.160 \, м, l = 20 \, м, K = 0,01 \, м^3/сутки $, имеем:	$f = 280 \, м^3 / сут$

	\bigskip
	
При дебите дрены $24 м^3/час$ условие \eqref{eq:debit} выполняется. Таким образом, \textbf{водозахватная способность дренажных скважин достаточна} для успешного водопонижения рассматриваемого участка.

\end{document} % конец документа

